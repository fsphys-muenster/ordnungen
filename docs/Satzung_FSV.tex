% Autor: Simon May; Maik Stappers
% Datum: 2014-12-26, Änderung 10.11.2015

%%% Pakete & Klassen
% Verwendung von KOMA-Script
\documentclass[
	% Papierformat
	a4paper,
	% Schriftgröße
	12pt,
	% einseitiges Layout
	oneside,
	% Abstand zwischen Absätzen statt Einrücken
	parskip=half-,
	% Papiergröße korrekt ins Dokument schreiben
	pagesize,
	% Linie unter der Kopfzeile
	headsepline,
	% deutsches Dokument (neue deutsche Rechtschreibung)
	german,
	ngerman
]{scrartcl}
\usepackage{iftex}
\ifLuaTeX
	% Einstellungen für Schriftart
	\usepackage{fontspec}
	\defaultfontfeatures{Ligatures=TeX}
	% Silbentrennung, sprachspezifische Einstellungen
	\usepackage{polyglossia}
	\setmainlanguage{german}
\else
	% Silbentrennung, sprachspezifische Einstellungen
	\usepackage{babel}
	% Mögliche darstellbare Zeichen (Umlaute, Sonderzeichen...)
	\usepackage[T1]{fontenc}
	% Zeichenkodierung der TeX-Datei
	\usepackage[utf8]{inputenc}
	% Schriftart
	\usepackage{lmodern}
	% führt Befehle für Sonderzeichen ein
	\usepackage{textcomp}
\fi
% Einige LaTeX-Bugs beheben
\usepackage{fixltx2e}
% Für Kopf-/Fußzeile etc.
\usepackage{scrpage2}
% Farben ermöglichen
\usepackage{xcolor}
% Paket für Bilder-Einbindung (EPS, PNG, JPG, PDF)
\usepackage{graphicx}
% Formatierung von Daten
\usepackage{datetime}
% Einstellungen für Aufzählungen
\usepackage{enumitem}
% "Schlaue" Anführungszeichen
\usepackage{csquotes}
% Seitenränder
\usepackage[
	left=2cm,
	right=2cm,
	top=2cm,
	bottom=1.3cm,
	includeheadfoot
]{geometry}
% Verlinkung, Querverweise können im PDF angeklickt werden
\usepackage[bookmarksnumbered, unicode]{hyperref}

\usepackage[deletedmarkup=sout]{changes} %Erlaubt das markieren von Änderungen
\usepackage{color} %Ändern von Farben
\setdeletedmarkup{\textcolor{red}{\sout{#1}}}
%%% Einstellungen
% Anführungszeichen automatisch umwandeln
\MakeOuterQuote{"}

\hypersetup{
	% Links/Verweise in PDF mit Kasten der Dicke 0.5pt versehen
	pdfborder={0 0 0.5},
	pdfauthor={Fachschaftsvertretung Physik der WWU Münster},
	pdftitle={Fachschaftsordnung der Fachschaft Physik der Westfälischen Wilhelms-Universität Münster},
	pdfkeywords={Fachschaftsvertretung, FSV, WWU, Münster, Satzung}
}

% Kopfzeile
\chead{\replaced{Fachschaftsordnung}{Satzung} der Fachschaft\deleted{svertretung} Physik der Westfälischen Wilhelms-Universität Münster}
\pagestyle{scrheadings}

% Darstellung von sections anpassen
\renewcommand{\thesection}{§~\arabic{section}}

% Befehl zur Darstellung von E-Mail-Adressen
\newcommand{\email}[1]{\href{mailto:#1}{\texttt{#1}}}

%%% Bestandteile des Titels
\titlehead{\vspace*{-2cm}
\centering
\includegraphics[width=0.8\textwidth]{Fachschaft_Physik_Logo_v2014_2.pdf}}
\title{\replaced{Fachschafsordnung}{Satzung}}
\subtitle{der Fachschaft\deleted{svertretung} Physik\\
der Westfälischen Wilhelms-Universität Münster}
\date{in der Fassung vom \formatdate{11}{11}{2011}}
\author{}

\begin{document}

% Titel mit Logo
\maketitle
\section*{Erläuterungen}
\begin{description}
	\item[\textcolor{green}{grüner Text}] Anmerkungen des Autors zur Erläuterung der Änderungen.
	\item[\textcolor{red}{\sout{roter Text}}] Löschungen
	\item[\textcolor{blue}{blauer Text}] Einfügungen 
\end{description}

\section*{\textcolor{green}{Anmerkungen}}

\textcolor{green}{Die Änderungen an der "Satzung der Fachschaftsvertrung  Physik" sollen die Arbeit im Ganzen etwas einfacher und strukturierter machen. Ich bin der Meinung, dass die Satzung wie sie momentan ist viele Fragen offen lässt und außerdem mit der neuen Satzung der Studierendenschaft nicht vereinbar ist. Diese Version ist öffentlich einsehbar und ich hoffe dass viele Fachschaftsvertreter der Physik sie lesen, damit wir diese Ordnung auf der konstituiereden Sitzung im Dezember beschließen können. Da diese Ordnung bei Github gehostet wird kann jeder von euch Änderungsvorschläge einbringen.\\
Ich hoffe dass diese Fachschaftsordnung, oder kurz FO, nicht zu viele Diskussionen verursachen wird. Falls jemand von euch der Meinug sein sollte diverse Passagen dieses Dokuments seien totaler Quatsch, so teilt mir das gerne mit, ich bin bereit über jeden Punkt hier zu diskutieren. \\ Diese FO muss auch nciht zwangsläufig auf der konstituiereden Sitzung beschlossen werden, jedoch muss sie spätestens zum Sommersemester beschlossen werden, falls dies nicht passiert haben wir keine FO, die Dinge außerhalb zu erlassender GOs regelt.\\
Zunächst aber eine kurze Erklärung warum viele der Änderungen in meinen Augen nötig sind. 
\begin{itemize}
	\item Die Studierendenschaft hat sich eine neue Satzung gegeben und dabei einige Punkte die Fachschaften betreffend geändert:
	\begin{enumerate}
		\item Eine FS darf keine Satzung haben (Durften wir bisher auch nicht). Jetzt dürfen wir nur noch eine "Fachschaftsordnung" haben
		\item Fachschaften haben keine Organe, das darf nur die Studierendenschaft, wir haben Gremien.
		\item Die Aufgaben der Fachschaft sind umfassend in der Satzung der Studierendenschaft beschrieben, daher taucht hier nix davon auf.
		\item Finanzer und Vorsitz des Fachschaftsrat dürfen nicht mehr die gleichen Personen sein. 
		\item Auf der FSV Sitzung sollten wir mindestens die geforderten Posten "Finanzen" und "Vorsitz" wählen, zusätzlich ist es sicherlich sinnvoll direkt jemanden für die Evaluation und die Koordination der Erstidinge zu wählen. Alles andere kann auf der FSV Sitzung geplant/besprochen/gewählt werden.
	\end{enumerate}
		\item Einige Aufgabenverteilung sind aus der GO des FSR rausgeflogen und stehen jetzt hier, weil diese Ordnung auch bei Neukonstituierung ihre Gültigkeit behält, eine GO muss neu beschlossen werden.
		\item §~6 bindet die Finanzer an entscheidungen des Fachschaftsrats bzw. der Fachschaftsvertretung, aktuell könnte ich meine privaten Einkäufe einreichen.
\end{itemize}
}
\section{Präambel}
Hiermit gibt sich die Fachschaftsvertretung Physik (FSV) der Westfälischen Wilhelms-Universi\-tät Münster (WWU) auf Grundlage von §~\replaced{41}{22} Abs.~\replaced{1}{4--5} der Satzung der Studierendenschaft der WWU eine Fahschaftsordnung (FO) im Rahmen der geltenden Gesetze und der Satzung der Studierendenschaft. Sie regelt ergänzend die Angelegenheiten der Fachschaft \deleted{und ihrer Organe}.

\deleted{Die offizielle Bezeichnung der Fachschaft ist zusätzlich \textbf{111} (§~19 Abs.~2 Satzung der Studierendenschaft).}

\section{\deleted{Mitgliedschaft}}
\textcolor{green}{Können wir leider nicht regeln, daher raus damit}
\begin{enumerate}
	\item \deleted{Gemäß §~xx Abs.~x der Satzung der Studierendenschaft ist jede/jeder Studierende Mitglied der Fachschaft Physik der WWU, wenn er/sie mit dem Hauptfach Physik immatrikuliert ist.}
	\item \deleted{Ausschließlich ordentliche Mitglieder gemäß §~2 Abs.~1 haben das aktive und passive Wahlrecht zur Fachschaftsvertretung. Gast- und Zweithörer haben kein Wahlrecht zur Fachschaftsvertretung Physik. Näheres regelt die Wahlordnung der Studierendenschaft.}
\end{enumerate}

\setcounter{section}{1}
\section{\replaced{Gremien}{Organe} der Fachschaft}
Die \replaced{Gremien}{Organe} der Fachschaft sind gemäß der Satzung der Studierendenschaft:
\begin{enumerate}
	\item die Fachschaftsvertretung (FSV) (§~\replaced{37}{22} Satzung der Studierendenschaft),
	\item der Fachschaftsrat (FSR) (§~\replaced{38}{23} Satzung der Studierendenschaft) und 
	\item die Fachschaftsvollversammlung (FVV) (§~\replaced{39}{24} Satzung der Studierendenschaft).
\end{enumerate}

\section{Fachschaftsvertretung}
Die Amtszeit, Zusammensetzung, Einberufung, Aufgaben und Beschlussfassung der FSV sind in §~\replaced{37}{22} der Satzung der Studierendenschaft geregelt. Näheres zur Wahl der Fachschaftsvertretungen regelt die Wahlordnung.

Ergänzend gibt sich die FSV die zusätzlichen Regelungen:
\begin{enumerate}
	\item Die FSV wählt auf ihrer konstituierenden Sitzung den Fachschaftsrat. Die Sitzung ist öffentlich. Der FSV wird empfohlen, nur Personen in den FSR zu wählen, die mindestens \replaced{im}{das} jeweilige laufende Semester aktiv in der Fachschaft mitgearbeitet haben.
	\item Die Wahl des FSR \added{findet unter Ausschluss der Öffentlichkeit statt und }erfolgt \replaced{durch}{mit} Handzeichen. Auf Antrag einer anwesenden wahlberechtigten Person muss eine geheime Wahl stattfinden. Es kann auch über die Liste als Ganzes abgestimmt werden.
	\item \replaced{Die Fachschaftsvertretung gibt sich eine Geschäftsordnung}{Gemäß §~22 Abs.~6 der Satzung der Studierendenschaft gibt sich die Fachschaftsvertretung Physik eine Geschäftsordnung.}
\end{enumerate}

\section{Aufgaben der Fachschaft}
\begin{enumerate}
	\item Die Aufgaben der Fachschaft sind in §~\replaced{36}{20} der Satzung der Studierendenschaft aufgeführt.
	\item \replaced{Zusätzlich wird eine Enge Kooperation mit der Fachschaft Geophysik angestrebt.}{Für die Studierenden im Studienfach Geophysik ist der FSR Geophysik Ansprechpartner. Eine ausführliche Kooperation mit dem FSR Geophysik wird angestrebt}.\textcolor{green}{Mit dem die Geophysik studis reden müssen hat bei uns in der FO nix zu suchen}
\end{enumerate}

\section{\added{Zusammensetzung des Fachschaftsrat}}
\begin{enumerate}
	\item \added{Der Fachschaftsrat ist nach §~38 der Satzung der Studierendenschaft das Exekutivgremium der Fachschaft.}
	\item \added{Der Fachschaftsrat gliedert sich in verschiedene Geschäftschäftsbereiche. Jeder Geschäftsbereich ist mit mindestens zwei Personen zu besetzen. Der Fachschaftsrat muss mindestens folgende Geschäftsbereiche aufweisen:}
	\begin{enumerate}
		\item \added{Vorsitz}
		\item \added{Finanzen}
		\item \added{Vorlesungsevaluation}
		\item \added{Erstsemesterarbeit}
	\end{enumerate}
	\item \added{Die Fachschaftsvertretung kann bei der Wahl des Fachschaftsrates weitere Geschäftsbereiche einrichten.}
	\item \added{Entsprechend §~38 Satzung der Studierendenschaft, dürfen die Geschäftsbereiche "Vorsitz" und "Finanzen" nicht mit den gleichen Personen besetzt werden.}
	\item \added{Der Geschäftsbereich "Vorsitz" besteht aus genau zwei Personen, der/dem Vorsitzenden des Fachschaftsrats und seinem/ihrem Stellvertreter*in.}\textcolor{green}{Ja, das muss gegendert werden...}
	\item \added{Ein FSR-Mitglied ist für die Verwaltung der Schlüssel zum Fachschaftsraum zuständig (Schlüsselwart)}.\textcolor{green}{Schlüsselwart ist kein eigener Geschäftsbereich, das kann irgendeiner der FSR'ler machen. Wegen des verschlossenen Schrankes würde ich die Finanzer vorschlagen}
\end{enumerate}

\section{Arbeit des Fachschaftsrats}
\deleted{§~23 der Satzung der Studierendenschaft regelt die Zusammensetzung des Fachschaftsrats (FSR). Zusätzlich gilt für die Arbeit des Fachschaftsrats Folgendes:} 
Der FSR nimmt die Aufgaben der Fachschaft entsprechend \deleted{der Geschäftsordnung der Fachschaftsvertretung Physik und} der Satzung der Studierendenschaft wahr. Ergänzend wird dem FSR aufgetragen, der FSV zu ihrer konstituierenden Sitzung einen Finanzbericht vorzulegen. Dieser Finanzbericht soll Auskunft über alle Einnahmen und Ausgaben des FSR während seiner Amtszeit geben.

\begin{itemize}
	\item \added{Der FSR gewährleistet eine angemessene Beratung der Studierenden und sorgt für Präsenzzeiten. Die Präsenzzeiten werden in einem Plan festgehalten, der am Anfang eines jeden Semesters aufzustellen ist. In der Vorlesungszeit ist ein Minimum von zwei Tagen in der Woche mit insgesamt mindestens vier Stunden Präsenzzeit zu gewährleisten. Die Mitglieder des FSR sorgen für den Präsenzdienst. Während des Präsenzdienstes soll ein umfangreiches Beratungsangebot garantiert werden.}
	\item \added{Der FSR sorgt durch Öffentlichkeitsarbeit für Transparenz und Anerkennung unter der Studierendenschaft. Dies wird unter anderem durch regelmäßige Information auf der Internetseite sowie durch Aushänge und Informationsveranstaltungen gewährleistet.}
	\item \added{Der FSR sorgt für eine umfassende Bereitstellung von studien- und prüfungsrelevantem Informationsmaterial, darunter die Ausleihe von Altklausuren und Prüfungsprotokollen. Die Ausleihe ist in einer separaten Anleitung geregelt; sie bildet keinen Teil der Fachschaftsordnungsordnung.}
\end{itemize}
\textcolor{green}{Gehört halt hier rein und nicht in die GO des FSR. Ansonsten steht das so in der FSR GO}

\section{\added{Fachschaftsfinanzen}}
\label{sec:Finanzen}
\textcolor{green}{Beschränkt die Finanzer zwar etwas, man kann aber zu Beginn des Jahres als FSR so etwas beschließen wie, für das SoFe haben wir ein Budget von 500€, für die O-Woche 1500€ und für das Wochenende 500€. So muss dafür nicht extra entschieden werden, es sei denn die Veranstaltung macht derbe Miese, aber dann sollte der Rat das aufjeden fall beschließen um die Finanzer abzusichern. Dern Rest kann man dann Situationsabhängig für anderen Kram verausgaben.}
\begin{enumerate}
	\item \added{Die Finanzmittel der Fachschaft Physik werden durch den AStA bewirtschaftet. Ausgaben werden durch die Finanzräte (Mitglieder des Geschäftsbereich "Finanzen") beantragt. (§~40 Abs.~2 Satzung der Studierendenschaft)}
	\item \added{Die Finanzräte dürfen nur nach vorherigem Beschluss eines Gremiums der Fachschaft Physik die Auszahlung der Finanzmittel beim AStA beantragen.}\label{Finanzen_2}
	\item \added{Bei Anträgen auf Auszahlung von Beträgen unter 20€ kann abweichend von §~6 Abs.~2 auf einen Beschluss eines Gremiums der Fachschaft Physik verzichtet werden.}
\end{enumerate}

\section{\added{Fachschaftsraum}}
\textcolor{green}{Gehört halt hier rein und nicht in die GO des FSR. Ansonsten steht das so in der FSR GO}
\begin{enumerate}
\item \added{Der FSR ist für die Ordnung und Sauberkeit des Fachschaftsraums zuständig.}
\item \added{Zu Beginn eines Semesters wird eine Reinigungsliste angelegt, auf der festgehalten wird, wer in welcher Woche den Fachschaftsraum zu säubern hat. Jedes FSR-Mitglied ist dazu angehalten, mindestens einmal im Semester diesen Dienst zu übernehmen.}
\item \added{Der Fachschaftsraum muss immer verschlossen sein, wenn niemand im Raum ist.}
\item  \added{Ein Schlüssel wird nur an diejenigen Personen ausgegeben, die diesen nachvollziehbar benötigen. Gleichzeitig wird erwartet, dass alle Mitglieder einen Präsenzdienst übernehmen, wenn ein Schlüssel ausgegeben wurde.}
\item \added{Der Schlüssel muss spätestens mit Austritt aus dem FSR zurückgegeben werden. Auf Antrag kann dieser in Ausnahmefällen auch an ehemalige Mitglieder des FSR ausgegeben werden. Es ist zu begründen, wofür dieser benötigt wird. Eine Abstimmung auf einer FSR-Sitzung ist dazu erforderlich.}

\end{enumerate}

\section{\added{E-Mail-Verteiler und Nutzergruppe}}
\textcolor{green}{Gehört halt hier rein und nicht in die GO des FSR. Ansonsten steht das so in der FSR GO}
\begin{enumerate}
	\item \added{Der FSR unterhält einen E-Mail-Verteiler: \textbf{\email{fsphys-l@listserv.uni-muenster.de}}. Dieser wird von Herrn Dr.\ Adam (\email{adamh@uni-muenster.de}) technisch verwaltet. E-Mails an die E-Mail-Adresse der Fachschaft (\email{fsphys@uni-muenster.de}) werden vom Verwalter nach einer Spam-Filterung an den internen Verteiler weitergeleitet. Die interne Kommunikation des FSR erfolgt über den Verteiler.}
	\item \added{Mitglied in dem E-Mail-Verteiler wird jedes Mitglied des FSR. Dem Verwalter wird jeweils mitgeteilt, wie er über einen Aufnahmeantrag zu entscheiden hat.}
	\item \added{Mit Austritt aus dem FSR endet die Mitgliedschaft im E-Mail-Verteiler. Alte Mitglieder können (mündlich) beantragen, weiterhin Mitglied im E-Mail-Verteiler zu bleiben. Der FSR entscheidet über den Antrag.}
	\item \added{Der FSR unterhält eine Nutzergruppe: \textbf{\texttt{p0fsphys}}. Diese wird von Herrn Dr. Berkemeier (\email{j.berkemeier@uni-muenster.de}) verwaltet. Die Mitgliedschaft ermöglicht den Zugang zum Gruppenlaufwerk und das Drucken auf dem Drucker im Fachschaftsraum.}
	\item \added{Mitglied in der Nutzergruppe wird jedes Mitglied des FSR. Anträge auf Aufnahme und Verlängerung der Mitgliedschaft in der Nutzergruppe erfolgen über das ZIV und werden an den Verwalter der Nutzergruppe weitergeleitet.}
	\item \added{Mit Austritt aus dem FSR erlischt die Mitgliedschaft in der Nutzergruppe. Nur aus triftigem Grund ist eine weitere Mitgliedschaft möglich. Dies muss in jedem Einzelfall vom FSR beschlossen werden.}
	\item \added{Der FSR beauftragt eine oder mehrere Personen mit der Betreuung und Dokumentation des E-Mail-Verteilers und der Nutzergruppe. Diese Personen sollen in regelmäßigen Abständen, z.\,B.\ einmal im Jahr, die aktuellen Mitglieder ermitteln und sind Ansprechpartner für die Verwalter.}
	\label{item:Ansprechpartner}
\end{enumerate}

\section{Fachschaftsvollversammlung}
\replaced{§~39}{§~24} der Satzung der Studierendenschaft regelt die Einberufung und Beschlussfähigkeit der Fachschaftsvollversammlung (FVV).

Für die FVV ist die Geschäftsordnung de\replaced{r}{s} Fachschaft\replaced{rats}{svertretung} Physik anzuwenden, insbesondere in Hinblick auf Redeleitung \deleted{, Rede-, Stimm- und Antragsrecht} und Protokollführung.

\section{Inkrafttreten}
\replaced{Die Fachschaftsordnung der Fachschaft Physik der Westfälischen Wilhelms Universität tritt durch Votum mit Zweidrittelmehrheit der Mitgliedern der Fachschaftsvertretung und durch öffentlichen Aushang am \formatdate{11}{11}{1111} in Kraft.}{Die Satzung der Fachschaftsvertretung Physik der WWU tritt durch schriftliches Votum mit Zweidrittelmehrheit der Mitglieder der Fachschaftsvertretung und durch öffentlichen Aushang am 1. August 2007 in Kraft.}

\deleted{Die Änderung der Satzung der Fachschaftsvertretung Physik der WWU tritt durch schriftliches Votum mit Zweidrittelmehrheit der Mitglieder der Fachschaftsvertretung und durch Veröffentlichung auf den Internetseiten der Fachschaft am 19. Januar 2011 in Kraft. Gleichzeitig tritt die Satzung vom 01. August 2007 außer Kraft.}

\deleted{Die Änderung der Satzung der Fachschaftsvertretung Physik der WWU tritt durch Votum mit Zweidrittelmehrheit der Mitglieder der Fachschaftsvertretung und durch öffentlichen Aushang am 17. Dezember 2014 in Kraft. Gleichzeitig tritt die Satzung vom 19. Januar 2011 außer Kraft.}

\end{document}
