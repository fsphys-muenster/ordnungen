%%% Pakete & Klassen
% Verwendung von KOMA-Script
\documentclass[
	% Papierformat
	a4paper,
	% Schriftgröße
	12pt,
	% einseitiges Layout
	oneside,
	% Abstand zwischen Absätzen statt Einrücken
	parskip=half-,
	% Papiergröße korrekt ins Dokument schreiben
	pagesize,
	% Linie unter der Kopfzeile
	headsepline,
	% deutsches Dokument (neue deutsche Rechtschreibung)
	german,
	ngerman
]{scrartcl}
\usepackage{iftex}
\ifLuaTeX
	% Einstellungen für Schriftart
	\usepackage{fontspec}
	% Silbentrennung, sprachspezifische Einstellungen
	\usepackage{polyglossia}
	\setmainlanguage{german}
\else
	% Silbentrennung, sprachspezifische Einstellungen
	\usepackage{babel}
	% Mögliche darstellbare Zeichen (Umlaute, Sonderzeichen...)
	\usepackage[T1]{fontenc}
	% Zeichenkodierung der TeX-Datei
	\usepackage[utf8]{inputenc}
	% Schriftart
	\usepackage{lmodern}
	% führt Befehle für Sonderzeichen ein
	\usepackage{textcomp}
\fi
% Einige LaTeX-Bugs beheben
\usepackage{fixltx2e}
% Für Kopf-/Fußzeile etc.
\usepackage{scrpage2}
% Farben ermöglichen
\usepackage{xcolor}
% Paket für Bilder-Einbindung (EPS, PNG, JPG, PDF)
\usepackage{graphicx}
% Formatierung von Daten
\usepackage{datetime}
% Einstellungen für Aufzählungen
\usepackage{enumitem}
% "Schlaue" Anführungszeichen
\usepackage{csquotes}
% Seitenränder
\usepackage[
	left=2cm,
	right=2cm,
	top=2cm,
	bottom=1.3cm,
	includeheadfoot
]{geometry}
% Verlinkung, Querverweise können im PDF angeklickt werden
\usepackage[bookmarksnumbered, unicode]{hyperref}

%%% Einstellungen
% Anführungszeichen automatisch umwandeln
\MakeOuterQuote{"}

\hypersetup{
	% Links/Verweise in PDF mit Kasten der Dicke 0.5pt versehen
	pdfborder={0 0 0.5},
	pdfauthor={Fachschaftsvertretung Physik der WWU Münster},
	pdftitle={Fachschaftsordnung der Fachschaft Physik der Westfälischen Wilhelms-Universität Münster},
	pdfkeywords={Fachschaftsvertretung, FSV, WWU, Münster, Satzung}
}

% Kopfzeile
\chead{Fachschaftsordnung der Fachschaft Physik der Westfälischen Wilhelms-Universität Münster}
\pagestyle{scrheadings}

% Darstellung von sections anpassen
\renewcommand{\thesection}{§~\arabic{section}}

% Befehl zur Darstellung von E-Mail-Adressen
\newcommand{\email}[1]{\href{mailto:#1}{\texttt{#1}}}

%%% Bestandteile des Titels
\titlehead{\vspace*{-2cm}
\centering
\includegraphics[width=0.8\textwidth]{Fachschaft_Physik_Logo_v2014_2.pdf}}
\title{Fachschaftsordnung}
\subtitle{der Fachschaft Physik\\
der Westfälischen Wilhelms-Universität Münster}
\date{in der Fassung vom \formatdate{11}{01}{2017}}
\author{}

\begin{document}

% Titel mit Logo
\maketitle

\section{Präambel}
Hiermit gibt sich die Fachschaftsvertretung Physik (FSV) der Westfälischen Wilhelms-Universi\-tät Münster (WWU) auf Grundlage von §~41 Abs.~1 der Satzung der Studierendenschaft der WWU eine Fachschaftsordnung (FO) im Rahmen der geltenden Gesetze und der Satzung der Studierendenschaft. Sie regelt ergänzend die Angelegenheiten der Fachschaft.

\setcounter{section}{1}
\section{Gremien der Fachschaft}
Die Gremien der Fachschaft sind gemäß der Satzung der Studierendenschaft:
\begin{enumerate}
	\item die Fachschaftsvertretung (FSV) (§~37 Satzung der Studierendenschaft),
	\item der Fachschaftsrat (FSR) (§~38 Satzung der Studierendenschaft) und 
	\item die Fachschaftsvollversammlung (FVV) (§~39 Satzung der Studierendenschaft).
\end{enumerate}

\section{Fachschaftsvertretung}
Die Amtszeit, Zusammensetzung, Einberufung, Aufgaben und Beschlussfassung der FSV sind in §~37 der Satzung der Studierendenschaft geregelt. Näheres zur Wahl der Fachschaftsvertretungen regelt die Wahlordnung.

Ergänzend gibt sich die FSV die zusätzlichen Regelungen:
\begin{enumerate}
	\item Die FSV wählt auf ihrer konstituierenden Sitzung den Fachschaftsrat. Die Sitzung ist öffentlich. Der FSV wird empfohlen, nur Personen in den FSR zu wählen, die mindestens im jeweilige laufende Semester aktiv in der Fachschaft mitgearbeitet haben.
	\item Die Wahl des FSR erfolgt durch Handzeichen. Auf Antrag einer anwesenden wahlberechtigten Person muss eine geheime Wahl stattfinden. Es kann auch über die Liste als Ganzes abgestimmt werden.
	\item Die Fachschaftsvertretung gibt sich eine Geschäftsordnung.
\end{enumerate}

\section{Aufgaben der Fachschaft}
\begin{enumerate}
	\item Die Aufgaben der Fachschaft sind in §~36 der Satzung der Studierendenschaft aufgeführt.
	\item Zusätzlich wird eine enge Kooperation mit der Fachschaft Geophysik angestrebt.
\end{enumerate}

\section{Zusammensetzung des Fachschaftsrat}
\begin{enumerate}
	\item Der Fachschaftsrat ist nach §~38 der Satzung der Studierendenschaft das Exekutivgremium der Fachschaft.
	\item Der Fachschaftsrat gliedert sich in verschiedene Geschäftschäftsbereiche. Jeder Geschäftsbereich ist mit mindestens zwei Personen zu besetzen. Der Fachschaftsrat muss mindestens folgende Geschäftsbereiche aufweisen:
	\begin{enumerate}
		\item Vorsitz
		\item Finanzen
		\item Vorlesungsevaluation
		\item Erstsemesterarbeit
	\end{enumerate}
	\item Die Fachschaftsvertretung kann bei der Wahl des Fachschaftsrates weitere Geschäftsbereiche einrichten.
	\item Entsprechend §~38 Satzung der Studierendenschaft dürfen die Geschäftsbereiche "Vorsitz" und "Finanzen" nicht mit der gleichen Person besetzt werden.
	\item Der Geschäftsbereich "Vorsitz" besteht aus genau zwei Personen, der/dem Vorsitzenden des Fachschaftsrats und seinem/ihrem Stellvertretenden.
	\item Ein FSR-Mitglied ist für die Verwaltung der Schlüssel zum Fachschaftsraum zuständig (Schlüsselwart).
\end{enumerate}

\section{Arbeit des Fachschaftsrats}

Der FSR nimmt die Aufgaben der Fachschaft entsprechend der Satzung der Studierendenschaft wahr. Ergänzend wird dem FSR aufgetragen, der FSV zu ihrer konstituierenden Sitzung einen Finanzbericht vorzulegen. Dieser Finanzbericht soll Auskunft über alle Einnahmen und Ausgaben des FSR während seiner Amtszeit geben.

\begin{itemize}
	\item Der FSR gewährleistet eine angemessene Beratung der Studierenden und sorgt für Präsenzzeiten. Die Präsenzzeiten werden in einem Plan festgehalten, der am Anfang eines jeden Semesters aufzustellen ist. In der Vorlesungszeit ist ein Minimum von zwei Tagen in der Woche mit insgesamt mindestens vier Stunden Präsenzzeit zu gewährleisten. Die Mitglieder des FSR sorgen für den Präsenzdienst. Während des Präsenzdienstes soll ein umfangreiches Beratungsangebot garantiert werden.
	\item Der FSR sorgt durch Öffentlichkeitsarbeit für Transparenz und Anerkennung unter der Studierendenschaft. Dies wird unter anderem durch regelmäßige Information auf der Internetseite sowie durch Aushänge und Informationsveranstaltungen gewährleistet.
	\item Protokolle über öffentliche Sitzungen des FSR werden zur Steigerung der Transparenz allen Physikstudierenden digital zugänglich gemacht
	\item Der FSR sorgt für eine umfassende Bereitstellung von studien- und prüfungsrelevantem Informationsmaterial, darunter die Ausleihe von Altklausuren und Prüfungsprotokollen. Die Ausleihe ist in einer separaten Anleitung geregelt; sie bildet keinen Teil der Fachschaftsordnungsordnung.
\end{itemize}


\section{Fachschaftsfinanzen}
\label{sec:Finanzen}

\begin{enumerate}
	\item Die Finanzmittel der Fachschaft Physik werden durch den AStA bewirtschaftet. Ausgaben werden durch die Finanzräte (Mitglieder des Geschäftsbereich "Finanzen") beantragt. (§~40 Abs.~2 Satzung der Studierendenschaft)
	\item Die Finanzräte dürfen nur nach vorherigem Beschluss eines Gremiums der Fachschaft Physik die Auszahlung der Finanzmittel beim AStA beantragen. Von dieser Regel ausgenommen sind Ausgaben für die Organisation und Durchführung des Sommerfests, der Orientierungswoche und der Erstsemester-Fahrt.
	\label{Finanzen_2}
	\item Bei Anträgen auf Auszahlung von Beträgen unter 20\,€ kann abweichend von §~6 Abs.~2 auf einen Beschluss eines Gremiums der Fachschaft Physik verzichtet werden.
\end{enumerate}

\section{Fachschaftsraum}

\begin{enumerate}
\item Der FSR ist für die Ordnung und Sauberkeit des Fachschaftsraums zuständig.
\item Zu Beginn eines Semesters wird eine Reinigungsliste angelegt, auf der festgehalten wird, wer in welcher Woche den Fachschaftsraum zu säubern hat. Jedes FSR-Mitglied ist dazu angehalten, mindestens einmal im Semester diesen Dienst zu übernehmen.
\item Der Fachschaftsraum muss immer verschlossen sein, wenn niemand im Raum ist.
\item Ein Schlüssel wird nur an diejenigen Personen ausgegeben, die diesen nachvollziehbar benötigen. Gleichzeitig wird erwartet, dass alle Mitglieder einen Präsenzdienst übernehmen, wenn ein Schlüssel ausgegeben wurde.
\item Der Schlüssel muss spätestens mit Austritt aus dem FSR zurückgegeben werden. Auf Antrag kann dieser in Ausnahmefällen auch an ehemalige Mitglieder des FSR ausgegeben werden. Es ist zu begründen, wofür dieser benötigt wird. Eine Abstimmung auf einer FSR-Sitzung ist dazu erforderlich.

\end{enumerate}

\section{E-Mail-Verteiler und Nutzergruppe}

\begin{enumerate}
	\item Der FSR unterhält einen E-Mail-Verteiler: \textbf{\email{fsphys-l@listserv.uni-muenster.de}}. Dieser wird von Herrn Dr.\ Adam (\email{adamh@uni-muenster.de}) technisch verwaltet. E-Mails an die E-Mail-Adresse der Fachschaft (\email{fsphys@uni-muenster.de}) werden vom Verwalter nach einer Spam-Filterung an den internen Verteiler weitergeleitet. Die interne Kommunikation des FSR erfolgt über den Verteiler.
	\item Mitglied in dem E-Mail-Verteiler wird jedes Mitglied des FSR. Dem Verwalter wird jeweils mitgeteilt, wie er über einen Aufnahmeantrag zu entscheiden hat.
	\item Mit Austritt aus dem FSR endet die Mitgliedschaft im E-Mail-Verteiler. Alte Mitglieder können (mündlich) beantragen, für ein weiteres Jahr Mitglied im E-Mail-Verteiler zu bleiben. Der FSR entscheidet über den Antrag.
	\item Der FSR unterhält eine Nutzergruppe: \textbf{\texttt{p0fsphys}}. Diese wird von Herrn Dr. Berkemeier (\email{j.berkemeier@uni-muenster.de}) verwaltet. Die Mitgliedschaft ermöglicht den Zugang zum Gruppenlaufwerk und das Drucken auf dem Drucker im Fachschaftsraum.
	\item Mitglied in der Nutzergruppe wird jedes Mitglied des FSR. Anträge auf Aufnahme und Verlängerung der Mitgliedschaft in der Nutzergruppe erfolgen über das ZIV und werden an den Verwalter der Nutzergruppe weitergeleitet.
	\item Mit Austritt aus dem FSR erlischt die Mitgliedschaft in der Nutzergruppe. Nur aus triftigem Grund ist eine weitere Mitgliedschaft möglich. Dies muss in jedem Einzelfall vom FSR beschlossen werden.
	\item Der FSR beauftragt eine oder mehrere Personen mit der Betreuung und Dokumentation des E-Mail-Verteilers und der Nutzergruppe. Diese Personen sollen in regelmäßigen Abständen, z.\,B.\ einmal im Jahr, die aktuellen Mitglieder ermitteln und sind Ansprechpartner für die Verwalter.
	\label{item:Ansprechpartner}
\end{enumerate}

\section{Fachschaftsvollversammlung}
§~39 der Satzung der Studierendenschaft regelt die Einberufung und Beschlussfähigkeit der Fachschaftsvollversammlung (FVV).

Für die FVV ist die Geschäftsordnung des Fachschaftrats Physik anzuwenden, insbesondere in Hinblick auf Redeleitung und Protokollführung.

\section{Inkrafttreten}
Die Fachschaftsordnung der Fachschaft Physik der Westfälischen Wilhelms Universität tritt durch Votum mit Zweidrittelmehrheit der Mitgliedern der Fachschaftsvertretung und durch öffentlichen Aushang am \formatdate{07}{01}{2016} in Kraft.

Die erste Änderung der Fachschaftsordnung der Fachschaft Physik der WWU tritt ddurch Votum mit Zweidrittelmehrheit der Mitgliedern der Fachschaftsvertretung und durch öffentlichen Aushang am \formatdate{12}{1}{2017} in Kraft. Gleichzeitig tritt die Fachschaftsordnung vom \formatdate{6}{1}{2016} außer Kraft.

\end{document}
