%%% Pakete & Klassen
% Verwendung von KOMA-Script
\documentclass[
	% Papierformat
	a4paper,
	% Schriftgröße
	12pt,
	% einseitiges Layout
	oneside,
	% Abstand zwischen Absätzen statt Einrücken
	parskip=half-,
	% Linie unter der Kopfzeile
	headsepline,
	% deutsches Dokument (neue deutsche Rechtschreibung)
	ngerman,
	% Kleinere Seitenränder
	DIV=15
]{scrartcl}
\usepackage{iftex}
\ifLuaTeX
	% Einstellungen für Schriftart
	\usepackage{fontspec}
	% Silbentrennung, sprachspezifische Einstellungen
	\usepackage{polyglossia}
	\setmainlanguage{german}
	\usepackage{selnolig}
\else
	% Silbentrennung, sprachspezifische Einstellungen
	\usepackage{babel}
	% Mögliche darstellbare Zeichen (Umlaute, Sonderzeichen…)
	\usepackage[T1]{fontenc}
	% Zeichenkodierung der TeX-Datei
	\usepackage[utf8]{inputenc}
	% Schriftart
	\usepackage{lmodern}
	% führt Befehle für Sonderzeichen ein
	\usepackage{textcomp}
\fi
% Besseres Schriftbild (Mikrotypographie)
\usepackage{microtype}
% Für Kopf-/Fußzeile etc.
\usepackage{scrlayer-scrpage}
% Farben ermöglichen
\usepackage{xcolor}
% Paket für Bilder-Einbindung (EPS, PNG, JPG, PDF)
\usepackage{graphicx}
% Größere Freiheiten bei Dateinamen mit \includegraphics
\usepackage{grffile}
% Formatierung von Daten
\usepackage[useregional]{datetime2}
% Einstellungen für Aufzählungen
\usepackage{enumitem}
% „Schlaue“ Anführungszeichen
\usepackage{csquotes}

% Verlinkung, Querverweise können im PDF angeklickt werden
\usepackage[unicode]{hyperref}

%%% Einstellungen
% Anführungszeichen automatisch umwandeln
\MakeOuterQuote{"}

\hypersetup{
	% Links/Verweise in PDF mit Kasten der Dicke 0.5pt versehen
	pdfborder={0 0 0.5},
}

% Darstellung von sections anpassen
\renewcommand{\thesection}{§~\arabic{section}}

\titlehead{\vspace*{-2cm}%
	\centering
	\includegraphics[width=0.8\textwidth]{fs-physik-logo_v2024}%
}

%%% Neue Befehle
% Befehl zur Darstellung von E-Mail-Adressen
\newcommand{\email}[1]{\href{mailto:#1}{\texttt{#1}}}
% Semantischer Befehl für starke Betonung


% PDF-Metadaten
\hypersetup{
	pdfauthor={Fachschaftsvertretung Physik der WWU Münster},
	pdftitle={Fachschaftsordnung der Fachschaft Physik der Westfälischen Wilhelms-Universität Münster},
	pdfkeywords={Fachschaftsvertretung, FSV, Physik, WWU, Westfälische Wilhelms-Universität, Münster, Fachschaftsordnung}
}
% Paket zur Nachverfolgung von Änderungen
%\usepackage{ulem} % package to strike out text
\usepackage[draft, commentmarkup=footnote]{changes} % change to final to hide comments
\setaddedmarkup{\textcolor{teal}{#1}}
\setdeletedmarkup{\textcolor{red}{\sout{#1}}}
\definechangesauthor[name=Maik, color=yellow]{MS}

% Kopfzeile
\ihead{Fachschaftsordnung der Fachschaft Physik der Westfälischen Wilhelms-Universität Münster}
\pagestyle{scrheadings}

% Bestandteile des Titels
\title{Fachschaftsordnung}
\subtitle{der Fachschaft Physik\\
der Westfälischen Wilhelms-Universität Münster}
\date{in der Fassung vom \replaced{\DTMdate{2019-01-09}} {\DTMdate{2018-01-10}}}
\author{}

\begin{document}

% Titel mit Logo
\maketitle

\section{Präambel}
Hiermit gibt sich die Fachschaft Physik der Westfälischen Wilhelms-Universität Münster (WWU), vertreten durch Fachschaftsvertretung Physik (FSV), auf Grundlage von §~38 Abs.~1 und §~42 Abs.~1 der Satzung der Studierendenschaft der WWU eine Fachschaftsordnung (FO) im Rahmen der geltenden Gesetze und der Satzung der Studierendenschaft.
Sie regelt ergänzend die Angelegenheiten der Fachschaft.

\section{Aufgaben der Fachschaft}
Die Aufgaben der Fachschaft sind in §~37 der Satzung der Studierendenschaft aufgeführt.
Zusätzlich wird eine enge Kooperation mit der Fachschaft Geophysik angestrebt.

\section{\added[id=MS, comment={Beschreibt den Paragrafen besser}]{Arbeit der} Fachschaftsvertretung}
\deleted[id=MS, comment={Den Absatz brauchen wir nicht in der FO}]{Die Aufgaben und Zusammensetzung der FSV sind in §~38 der Satzung der Studierendenschaft geregelt.
Näheres zur Wahl der Fachschaftsvertretungen regelt die Wahlordnung.
Ergänzend gibt sich die FSV Physik die zusätzlichen Regelungen:}
\begin{enumerate}
	\item \deleted[id=MS, comment={Das ist ne Empfehlung die im Zweifel eh niemanden interessiert, daher raus.}]{Der FSV wird empfohlen, nur Personen in den FSR zu wählen, die mindestens im jeweilig laufenden Semester aktiv in der Fachschaft mitgearbeitet haben.}
	\item \deleted[id=MS, comment={Das wird weiter unten noch geregelt.}]{Die Wahl des FSR erfolgt durch Handzeichen.
	Auf Antrag einer anwesenden wahlberechtigten Person muss eine geheime Wahl stattfinden.
	Es kann auch über eine Liste als Ganzes abgestimmt werden.}
	\item \deleted[id=MS, comment={Kommt später noch}]{Ist bei Personenwahlen nur eine Person oder bei Listenwahlen nur eine Liste aufgestellt, erfolgt die Wahl nach den Regeln für Abstimmungen.
	Fällt die Abstimmung positiv aus, ist die Person oder Liste gewählt; andernfalls ist sie nicht gewählt.}
\end{enumerate}
\begin{enumerate}
	\item \added{Die FSV wählt und kontrolliert den Fachschaftsrat}
	\item \added[id=MS, comment={Es FSV Vorsitzender ist immer sinnvoll}]{Die FSV wählt auf ihrer konstituierenden Sitzung aus ihrer Mitte durch Personenwahl gemäß §~7 Abs.~7 der Satzung der Studierendenschaft einzeln eine*n Vorsitzende*n und eine*n stellvertretende*n Vorsitzende*n.}
	\item \added[id=MS, comment={Eine monatliche FSV Sitzung gibt uns die Möglichkeit die Arbeit der FS zu evaluieren, und auf Änderungen/Probleme kurzfristig zu reagieren. Eine Sitzung kann im Zweifel auch nur aus "Hallo und Tschüss" bestehen.}]{Mindestens einmal pro Monat findet eine FSV Sitzung statt, hierzu lädt der Vorsitz der FSV mindestens eine Woche im voraus alle Mitlgieder der FSV ein.}
\end{enumerate}



\section{\replaced[id=MS, comment={Beschreibt den Paragrafen besser}]{Wahl des Fachschaftsrates (FSR)}{Zusammensetzung des Fachschaftsrats}}
\deleted[id=MS, comment={Der Satz ist unnötig.}]{Der FSR ist nach §~39 der Satzung der Studierendenschaft das Exekutivgremium der Fachschaft.}
\begin{enumerate}
	\item \added[id=MS, comment={Beschreibt das Wahlverfahren}]{\label{abs:wahl1} Die FSV legt die zu besetzenden Geschäftsbereiche des FSR und ihre Größe mit einfacher Mehrheit fest. Daraufhin wählt sie die Mitglieder des FSR durch Personenwahl gemäß §~7 Abs.~7 der Satzung der Studierendenschaft der WWU mit der Maßgabe, dass der dritte Wahlgang entfällt.}
	\item \label{abs:geschaeftsbereiche} Der FSR muss mindestens folgende Geschäftsbereiche aufweisen:
	\begin{enumerate}
		\item Vorsitz
		\item Finanzen \deleted[id=MS, comment={Müssen wir hier nicht stehen haben}]{(gemäß §~39 Abs.~2 der Satzung der Studierendenschaft)}
		\item Vorlesungsevaluation
		\item \deleted[id=MS, comment={Wir machen da heute schon 3 Bereiche raus, also sollten wir das nicht als Pflicht festlegen.}]{Erstsemesterarbeit}
	\end{enumerate}
	\deleted[id=MS, comment={Geschäftsbreiche können so groß sein wie wir das wollen.}]{Jeder Geschäftsbereich sollte mit mindestens zwei Personen besetzt werden.}
	\item \deleted[id=MS, comment={Ohne diesen Absatz könnten wir auch eine 3er Spitze wählen.}]{Der Geschäftsbereich 'Vorsitz' besteht aus genau zwei Personen, der/dem Vorsitzenden des FSR und seinem/ihrem Stellvertretenden.}
	\item \deleted[id=MS, comment={Das kommt weiter unten noch}]{Ein FSR-Mitglied ist für die Verwaltung der Schlüssel zum Fachschaftsraum zuständig (Schlüsselwart).}
\end{enumerate}
\begin{enumerate}
	\setcounter{enumi}{2}
	\item \added[id=MS, comment={Erlaubt es Leute ohne "Aufgabe" in den FSR zu wählen. Damit das nicht missbraucht werden kann, besteht hier die möglichkeit gegen die Kandidaten zu stimmen.}]{Die FSV kann weitere Mitglieder ohne Zuordnung zu einem Geschäftsbereich in den FSR wählen. Bei der Wahl dieser zusätzlichen Mitglieder des FSR haben die Mitglieder der FSV, abweichend von §~7 Abs.~1 der Satzung der Studierendenschaft, zusätzlich die Möglichkeit gegen den Kandidaten zu stimmen. Gewählt ist wer mehr Ja als Nein stimmen auf sich vereint.}
	\item \added[id=MS, comment={Das Stand sonst weiter oben.}]{Die Wahl des FSR erfolgt durch Handzeichen.
		Auf Antrag einer anwesenden wahlberechtigten Person muss eine geheime Wahl stattfinden.
		Es kann auch über eine Liste als Ganzes abgestimmt werden.}
	\item \added[id=MS, comment={Erlaubt das Ablehnen einer Liste, falls nur eine Liuste zur Wahl steht}]{Ist bei Listenwahlen nur eine Liste aufgestellt, können die Mitglider der FSV, abweichend von §~7 Abs.~1 der Satzung der Studierendenschaft, zusätzlich gegen die Vorschlagsliste stimmen.
	Die Liste ist gewählt, wenn sie mehr Ja als Nein stimmen auf sich vereint.}
	\item \added[id=MS, comment={Ermöglicht der FSV mehr Möglichkeiten den FSR zu wählen. Diese Änderungen gelten erst für die nächste Legislaturperiode. Damit dieses Verfahren angewandt wird, muss die FSV das mit einfacher Mehrheit beschließen, es können sich also nicht einzelne an der FAV vorbei den FSR bauen wie sie wollen.}]{Abweichend von Abs.~\ref{abs:wahl1} kann die FSV entsprechend §~42 Abs.~4 Nummer~2 der Satzung der Studierendenschaft beschließen, dass der Vorsitz des FSR die zu besetzenden Geschäftsbereiche und ihre Größe festlegt. In diesem Fall erhält der Vorsitz des FSR ein Vorschlagrecht für die Besetzung der Geschäftsbereiche. Abs.~\ref{abs:geschaeftsbereiche} bleibt hiervon unberührt.}
	\item \added[id=MS, comment={Ermöglicht der FSV mehr Möglichkeiten den FSR zu wählen. Diese Änderungen gelten erst für die nächste Legislaturperiode. Damit dieses Verfahren angewandt wird, muss die FSV das mit einfacher Mehrheit beschließen, es können sich also nicht einzelne an der FAV vorbei den FSR bauen wie sie wollen.}]{Die FSV kann entsprechend §~42 Abs.~4 Nummer~1 der Satzung der Studierendenschaft beschließen, dass der Vorsitz Richtlinien für die Tätigkeit der weiteren Mitglieder des FSR erlässt und	damit die weiteren Mitglieder ihre Tätigkeiten auch im Rahmen dieser Richtlinien wahrnehmen.}
\end{enumerate}

\section{Arbeit des Fachschaftsrats}
\begin{enumerate}
	\item Dem FSR wird aufgetragen, der FSV zu ihrer konstituierenden Sitzung einen Finanzbericht vorzulegen.
	Dieser Finanzbericht soll Auskunft über alle Einnahmen und Ausgaben des FSR während seiner Amtszeit geben.
	\item Der FSR gewährleistet eine angemessene Beratung der Studierenden und sorgt für Präsenzzeiten.
	Die Präsenzzeiten werden in einem Plan festgehalten, der am Anfang eines jeden Semesters aufzustellen ist.
	In der Vorlesungszeit ist ein Minimum von zwei Tagen in der Woche mit insgesamt mindestens vier Stunden Präsenzzeit zu gewährleisten.
	Die Mitglieder des FSR sorgen für den Präsenzdienst.
	Während des Präsenzdiensts soll ein umfangreiches Beratungsangebot garantiert werden.
	\item Der FSR sorgt durch Öffentlichkeitsarbeit für Transparenz und Anerkennung unter der Studierendenschaft.
	Dies wird unter anderem durch regelmäßige Information auf der Internetseite sowie durch Aushänge und Informationsveranstaltungen gewährleistet.
	\item Der FSR sorgt für eine umfassende Bereitstellung von studien- und prüfungsrelevantem Informationsmaterial, darunter die Ausleihe von Altklausuren und Prüfungsprotokollen.
	Die Ausleihe ist in einer separaten Anleitung geregelt; sie bildet keinen Teil der Fachschaftsordnung.
	\item Liegt kein anderslautender Beschluss des FSR vor, so finden Sitzungen des FSR während der Vorlesungszeit jeden Mittwoch um 18:00~Uhr \deleted[id=MS, comment={Wir haben schon lange keine Sitzung mehr im FS Raum gemacht.}]{im Fachschaftsraum} statt.
	Eine Einladung zu diesen Sitzungen ist nicht erforderlich.
\end{enumerate}

\section{Fachschaftsfinanzen}
\label{sec:finanzen}
\begin{enumerate}
	\item \label{item:finanzen_beschluss}
Die Finanzräte dürfen nur dann Finanzmittel beim AStA beantragen, wenn zuvor ein Gremium der Fachschaft Physik die Ausgaben bewilligt hat.
	Von dieser Regel ausgenommen sind Ausgaben für die Durchführung des Sommerfests, der Orientierungswoche und der Erstsemester-Fahrt.
	\item Bei Ausgaben von Beträgen unter 10\,€ kann abweichend von \ref{sec:finanzen} Abs.~\ref{item:finanzen_beschluss} auf einen Beschluss eines Gremiums der Fachschaft Physik verzichtet werden.
	\item Die Finanzräte erstatten dem Fachschaftsrat \replaced[id=MS, comment={Die FSV sollte auch wissen was die Finanzen der FSR machen. Auch sollten alle Studenten zumindest nachfragen/nachlesen können was die Finazen so machen.}]{und der Fachschaftsvertretung monatlich Bericht über die Fachschaftsfinazen. Der Finanzbericht ist öffentlich.}{monatlich über die Ausgaben des vergangen Monats Bericht.}
\end{enumerate}

\section{Fachschaftsraum}
\begin{enumerate}
	\item Der FSR ist für die Ordnung und Sauberkeit des Fachschaftsraums zuständig.
	\item Zu Beginn eines Semesters wird eine Reinigungsliste angelegt, auf der festgehalten wird, wer in welcher Woche den Fachschaftsraum zu säubern hat.
	Jedes FSR-Mitglied ist dazu angehalten, mindestens einmal im Semester diesen Dienst zu übernehmen.
	\item Der Fachschaftsraum muss immer verschlossen sein, wenn niemand im Raum ist.
	\item Ein Schlüssel wird nur an diejenigen Personen ausgegeben, die diesen nachvollziehbar benötigen.
	Gleichzeitig wird erwartet, dass alle Mitglieder einen Präsenzdienst übernehmen, wenn ein Schlüssel ausgegeben wurde.
	\item Der Schlüssel muss spätestens mit Austritt aus dem FSR zurückgegeben werden.
	\item Auf Antrag kann der Schlüssel in Ausnahmefällen auch an ehemalige Mitglieder des FSR ausgegeben werden.
	Es ist zu begründen, wofür \added{und wie lange} dieser benötigt wird.
	\replaced[id=MS, comment={Klarstellung des Verfahrens}]{Der FSR entscheidet mit einfacher Mehrheit über den Antrag.}{Eine Abstimmung auf einer FSR-Sitzung ist dazu erforderlich.}
	\item \added[id=MS, comment={Selbsterklärend}]{Der Fachschaftsraum und der Fachschafts-PC dient der Arbeit des FSR. Die Arbeit des FSR darf nicht durch nicht-fachschaftsbezogene Tätigkeiten im Fachschaftsraum beeinträchtigt werden.}
	\item \added[id=MS, comment={Selbsterklärend}]{Der Fachschaftsraum steht allen Fachschaftsratsmitgliedern zur Verfügung. Besonderen Stellenwert hat dabei die Nutzung im Rahmen von Aufgaben des Fachschaftsrats. Über die Dauer der aktiven Nutzung ist es deren aktives Bestreben, Studierende zu beraten. Dies ist durch das 'Fachschaft geöffnet'-Schild sowie durch eingeschaltetes Licht zu kennzeichnen. Außerhalb eines 2-Stunden-Rahmens um die Öffnungszeiten des Gebäudes sowie im Rahmen der Arbeit von Berufungskommissionen kann davon abgewichen werden. Nach Beschluss des FSR kann hiervon ebenfalls abgewichen werden.}
\end{enumerate}

\section{\deleted[id=MS, comment={Dieser Absatz regelt nur noch die p0fsphys gruppe}]{E-Mail-Verteiler und }Nutzergruppe}
\begin{enumerate}
	\item \deleted[id=MS, comment={kommt weiter unten wieder}]{Der FSR unterhält einen E-Mail-Verteiler: \email{fsphys-l@listserv.uni-muenster.de}.
	Die interne Kommunikation des FSR erfolgt über den Verteiler.}
	\item \deleted[id=MS, comment={kommt weiter unten wieder}]{Mitglied in dem E-Mail-Verteiler wird jedes Mitglied des FSR.
	Die Aufnahme weiterer Personen ist möglich.}
	\item \deleted[id=MS, comment={kommt weiter unten wieder}]{Mit Austritt aus dem FSR endet die Mitgliedschaft im E-Mail-Verteiler.
	Ehemalige Mitglieder des E-Mail-Verteilers können (auch mündlich) beantragen, für ein weiteres Jahr Mitglied im E-Mail-Verteiler zu sein.
	Mehrfachantrag ist möglich.
	Der FSR entscheidet über den Antrag.}
	\item Der FSR unterhält eine Nutzergruppe: \texttt{p0fsphys}.
	\deleted[id=MS, comment={Der FO kann egal sein wer die Gruppe verwaltet}]{Diese wird von Herrn Dr.~Berkemeier (\email{j.berkemeier@uni-muenster.de}) verwaltet.}
	Die Mitgliedschaft ermöglicht u.\,a.\ den Zugang zum Gruppenlaufwerk und das Drucken auf dem Drucker im Fachschaftsraum.
	\item Mitglied in der Nutzergruppe wird jedes Mitglied des FSR.
	Anträge auf Aufnahme und Verlängerung der Mitgliedschaft in der Nutzergruppe erfolgen über das Zentrum für Informationsverarbeitung (ZIV) und werden an den Verwalter der Nutzergruppe weitergeleitet.
	\item Mit Austritt aus dem FSR erlischt die Mitgliedschaft in der Nutzergruppe.
	Nur aus triftigem Grund ist eine weitere Mitgliedschaft möglich.
	Dies muss in jedem Einzelfall vom FSR beschlossen werden.
	\item Der FSR beauftragt eine oder mehrere Personen mit der Betreuung und Dokumentation \deleted[id=MS, comment={Nutzergruppe nud E-Mail-Verteiler werden heute schon von unterschiedlichen Leuten verwaltet.}]{des E-Mail-Verteilers und }der Nutzergruppe.
	Diese Personen sollen in regelmäßigen Abständen, z.\,B.\ einmal im Jahr, die aktuellen Mitglieder ermitteln.
	Die Beauftragten für die Nutzergruppe sind Ansprechpartner für deren Verwalter.
\end{enumerate}

\section{\added[id=MS, comment={Dieser Paragraph regelt die kommunikation innnerhalb der FAchschaft und ist eigentlich selbsterklärend}]{Interne Kommunikation}}
\begin{enumerate}
	\item \added{Der FSR unterhält einen E-Mail-Verteiler: \email{fsphys-l@listserv.uni-muenster.de}.}
	\item \added{er FSR unterhält einen Messengerdienst, der universitätsintern gehostet wird.}
	\item \added{Jedes Mitglied des FSR und jedes Mitglied der FSV wird in den E-Mail-Verteiler aufgenommen und erhält Zugriff auf den Messenger.}
	\item \added{Weitere Personen können in den Verteiler aufgenommen werden und Zugriff auf den Messenger erhalten. Dies muss in jedem Einzelfall vom FSR mit einfacher Mehrheit beschlossen werden.}
	\item \added{Mit Ende der Mitgliedschaft in FSR oder FSV endet auch die Mitgliedschaft im E-Mail-Verteiler und der Zugriff auf den Messenger.}
	\item \added{Ehemalige Mitglieder können (auch mündlich) beantragen, für ein weiteres Jahr Mitglied im E-Mail-Verteiler zu sein, und Zugriff auf den Messenger zu erhalten.
		Mehrfachantrag ist möglich.
		Der FSR entscheidet mit einfacher Mehrheit über den Antrag.}
	\item \added{Für die Interne Fachschaftskommunikation dürfen nur Universitätsinterne Kommunikationsdienste verwendet werden.}
\end{enumerate}

\section{Fachschaftsvollversammlung}
Für die Fachschaftsvollversammlung~(FVV) ist die Geschäftsordnung des Fachschaftsrats Physik anzuwenden, insbesondere im Hinblick auf Redeleitung und Protokollführung.

\section{Inkrafttreten}
Die Fachschaftsordnung der Fachschaft Physik der Westfälischen Wilhelms-Universität Münster wird durch Votum mit Zweidrittelmehrheit der Mitglieder der Fachschaftsvertretung beschlossen und tritt durch öffentlichen Aushang am \replaced{\DTMdate{2019-01-09}}{\DTMdate{2018-01-10}} am Folgetag in Kraft.

\end{document}
