%%% Pakete & Klassen
% Verwendung von KOMA-Script
\documentclass[
	% Papierformat
	a4paper,
	% Schriftgröße
	12pt,
	% einseitiges Layout
	oneside,
	% Abstand zwischen Absätzen statt Einrücken
	parskip=half-,
	% Linie unter der Kopfzeile
	headsepline,
	% deutsches Dokument (neue deutsche Rechtschreibung)
	ngerman,
	% Kleinere Seitenränder
	DIV=15
]{scrartcl}
\usepackage{iftex}
\ifLuaTeX
	% Einstellungen für Schriftart
	\usepackage{fontspec}
	% Silbentrennung, sprachspezifische Einstellungen
	\usepackage{polyglossia}
	\setmainlanguage{german}
	\usepackage{selnolig}
\else
	% Silbentrennung, sprachspezifische Einstellungen
	\usepackage{babel}
	% Mögliche darstellbare Zeichen (Umlaute, Sonderzeichen…)
	\usepackage[T1]{fontenc}
	% Zeichenkodierung der TeX-Datei
	\usepackage[utf8]{inputenc}
	% Schriftart
	\usepackage{lmodern}
	% führt Befehle für Sonderzeichen ein
	\usepackage{textcomp}
\fi
% Besseres Schriftbild (Mikrotypographie)
\usepackage{microtype}
% Für Kopf-/Fußzeile etc.
\usepackage{scrlayer-scrpage}
% Farben ermöglichen
\usepackage{xcolor}
% Paket für Bilder-Einbindung (EPS, PNG, JPG, PDF)
\usepackage{graphicx}
% Größere Freiheiten bei Dateinamen mit \includegraphics
\usepackage{grffile}
% Formatierung von Daten
\usepackage[useregional]{datetime2}
% Einstellungen für Aufzählungen
\usepackage{enumitem}
% „Schlaue“ Anführungszeichen
\usepackage{csquotes}

% Verlinkung, Querverweise können im PDF angeklickt werden
\usepackage[unicode]{hyperref}

%%% Einstellungen
% Anführungszeichen automatisch umwandeln
\MakeOuterQuote{"}

\hypersetup{
	% Links/Verweise in PDF mit Kasten der Dicke 0.5pt versehen
	pdfborder={0 0 0.5},
}

% Darstellung von sections anpassen
\renewcommand{\thesection}{§~\arabic{section}}

\titlehead{\vspace*{-2cm}%
	\centering
	\includegraphics[width=0.8\textwidth]{fs-physik-logo_v2024}%
}

%%% Neue Befehle
% Befehl zur Darstellung von E-Mail-Adressen
\newcommand{\email}[1]{\href{mailto:#1}{\texttt{#1}}}
% Semantischer Befehl für starke Betonung


% PDF-Metadaten
\hypersetup{
	pdfauthor={Fachschaftsvertretung Physik der WWU Münster},
	pdftitle={Fachschaftsordnung der Fachschaft Physik der Westfälischen Wilhelms-Universität Münster},
	pdfkeywords={Fachschaftsvertretung, FSV, Physik, WWU, Westfälische Wilhelms-Universität, Münster, Fachschaftsordnung}
}
% Paket zur Nachverfolgung von Änderungen
\usepackage[draft]{changes} % change to final to hide comments
\definechangesauthor[name=Maik, color=green]{MS}

% Kopfzeile
\ihead{Fachschaftsordnung der Fachschaft Physik der Westfälischen Wilhelms-Universität Münster}
\pagestyle{scrheadings}

% Bestandteile des Titels
\title{Fachschaftsordnung}
\subtitle{der Fachschaft Physik\\
der Westfälischen Wilhelms-Universität Münster}
\date{in der Fassung vom \replaced{\DTMdate{2019-01-16}} {\DTMdate{2018-01-10}}}
\author{}

\begin{document}

% Titel mit Logo
\maketitle

\section{Präambel}
Hiermit gibt sich die Fachschaft Physik der Westfälischen Wilhelms-Universität Münster (WWU), vertreten durch Fachschaftsvertretung Physik (FSV), auf Grundlage von §~38 Abs.~1 und §~42 Abs.~1 der Satzung der Studierendenschaft der WWU eine Fachschaftsordnung (FO) im Rahmen der geltenden Gesetze und der Satzung der Studierendenschaft.
Sie regelt ergänzend die Angelegenheiten der Fachschaft.

\section{Aufgaben der Fachschaft}
Die Aufgaben der Fachschaft sind in §~37 der Satzung der Studierendenschaft aufgeführt.
Zusätzlich wird eine enge Kooperation mit der Fachschaft Geophysik angestrebt.

\section{\added{Arbeit der }Fachschaftsvertretung}
\deleted{Die Aufgaben und Zusammensetzung der FSV sind in §~38 der Satzung der Studierendenschaft geregelt.
Näheres zur Wahl der Fachschaftsvertretungen regelt die Wahlordnung.
Ergänzend gibt sich die FSV Physik die zusätzlichen Regelungen:}
\begin{enumerate}
	\item \deleted{Der FSV wird empfohlen, nur Personen in den FSR zu wählen, die mindestens im jeweilig laufenden Semester aktiv in der Fachschaft mitgearbeitet haben.}
	\item \deleted{Die Wahl des FSR erfolgt durch Handzeichen.
	Auf Antrag einer anwesenden wahlberechtigten Person muss eine geheime Wahl stattfinden.
	Es kann auch über eine Liste als Ganzes abgestimmt werden.}
	\item \deleted{Ist bei Personenwahlen nur eine Person oder bei Listenwahlen nur eine Liste aufgestellt, erfolgt die Wahl nach den Regeln für Abstimmungen.
	Fällt die Abstimmung positiv aus, ist die Person oder Liste gewählt; andernfalls ist sie nicht gewählt.}
\end{enumerate}
\begin{enumerate}
	\item \added{Die FSV wählt und kontrolliert den Fachschaftsrat}
	\item \added{Die FSV wählt auf ihrer konstituierenden Sitzung aus ihrer Mitte durch Personenwahl gemäß §~7 Abs.~7 der Satzung der Studierendenschaft einzeln eine*n Vorsitzende*n und eine*n stellvertretende*n Vorsitzende*n.}
	\item \added{Mindestens einmal pro Monat findet eine FSV Sitzung statt, hierzu lädt der Vorsitz der FSV mindestens eine Woche im voraus alle Mitlgieder der FSV ein.}
\end{enumerate}



\section{\replaced{Wahl des Fachschaftsrates (FSR)}{Zusammensetzung des Fachschaftsrats}}
\deleted{Der FSR ist nach §~39 der Satzung der Studierendenschaft das Exekutivgremium der Fachschaft.}
\begin{enumerate}
	\item \replaced{\label{abs:wahl1} Die FSV legt die zu besetzenden Geschäftsbereiche des FSR und ihre Größe fest. Daraufhin wählt sie die Mitglieder des FSR durch Personenwahl gemäß §~7 Abs.~7 der Satzung der Studierendenschaft der WWU mit  der  Maßgabe,  dass  der  dritte  Wahlgang  entfällt.}{Der FSR muss mindestens folgende Geschäftsbereiche aufweisen:}
	\item \added{\label{abs:geschaeftsbereiche} Der FSR muss mindestens folgende Geschäftsbereiche aufweisen:}
	\begin{enumerate}
		\item Vorsitz
		\item Finanzen \deleted{(gemäß §~39 Abs.~2 der Satzung der Studierendenschaft)}
		\item \deleted{Vorlesungsevaluation}
		\item \deleted{Erstsemesterarbeit}
	\end{enumerate}
	\deleted{Jeder Geschäftsbereich sollte mit mindestens zwei Personen besetzt werden.}
	\item \deleted{Der Geschäftsbereich 'Vorsitz' besteht aus genau zwei Personen, der/dem Vorsitzenden des FSR und seinem/ihrem Stellvertretenden.}
	\item \deleted{Ein FSR-Mitglied ist für die Verwaltung der Schlüssel zum Fachschaftsraum zuständig (Schlüsselwart).}
\end{enumerate}
\begin{enumerate}
	\setcounter{enumi}{2}
	\item \added{Die FSV kann weitere Mitglieder ohne Zuordnung zu einem Geschäftsbereich in den FSR wählen.}
	\item \added{Die Wahl des FSR erfolgt durch Handzeichen.
		Auf Antrag einer anwesenden wahlberechtigten Person muss eine geheime Wahl stattfinden.
		Es kann auch über eine Liste als Ganzes abgestimmt werden.}
	\item \added{Ist bei Listenwahlen nur eine Liste aufgestellt, erfolgt die Wahl abweichend von §~7 Abs.~1 der Satzung der Studierendenschaft nach den Regularien für Abstimmung.  
		Fällt die Abstimmung positiv aus, ist die Liste gewählt; andernfalls ist sie nicht gewählt.}
	\item \added{Abweichend von Abs.~\ref{abs:wahl1} kann die FSV entsprechend §~42 Abs.~4 Nummer~2 der Satzung der Studierendenschaft beschließen, dass der Vorsitz des FSR die zu besetzenden Geschäftsbereiche und ihre Größe festlegt. In diesem Fall erhält der Vorsitz des FSR ein Vorschlagrecht für die Besetzung der Geschäftsbereiche. Abs.~\ref{abs:geschaeftsbereiche} bleibt hiervon unberührt.}
	\item \added{Die FSV kann entsprechend §~42 Abs.~4 Nummer~1 der Satzung der Studierendenschaft beschließen, dass der Vorsitz Richtlinien für die Tätigkeit der weiteren Mitglieder des FSR erlässt und	damit die weiteren Mitglieder ihre Tätigkeiten auch im Rahmen dieser Richtlinien wahrnehmen.}
\end{enumerate}

\section{Arbeit des Fachschaftsrats}
\begin{enumerate}
	\item Dem FSR wird aufgetragen, der FSV zu ihrer konstituierenden Sitzung einen Finanzbericht vorzulegen.
	Dieser Finanzbericht soll Auskunft über alle Einnahmen und Ausgaben des FSR während seiner Amtszeit geben.
	\item Der FSR gewährleistet eine angemessene Beratung der Studierenden und sorgt für Präsenzzeiten.
	Die Präsenzzeiten werden in einem Plan festgehalten, der am Anfang eines jeden Semesters aufzustellen ist.
	In der Vorlesungszeit ist ein Minimum von zwei Tagen in der Woche mit insgesamt mindestens vier Stunden Präsenzzeit zu gewährleisten.
	Die Mitglieder des FSR sorgen für den Präsenzdienst.
	Während des Präsenzdiensts soll ein umfangreiches Beratungsangebot garantiert werden.
	\item Der FSR sorgt durch Öffentlichkeitsarbeit für Transparenz und Anerkennung unter der Studierendenschaft.
	Dies wird unter anderem durch regelmäßige Information auf der Internetseite sowie durch Aushänge und Informationsveranstaltungen gewährleistet.
	\item Der FSR sorgt für eine umfassende Bereitstellung von studien- und prüfungsrelevantem Informationsmaterial, darunter die Ausleihe von Altklausuren und Prüfungsprotokollen.
	Die Ausleihe ist in einer separaten Anleitung geregelt; sie bildet keinen Teil der Fachschaftsordnung.
	\item Liegt kein anderslautender Beschluss des FSR vor, so finden Sitzungen des FSR während der Vorlesungszeit jeden Mittwoch um 18:00~Uhr \deleted{im Fachschaftsraum} statt.
	Eine Einladung zu diesen Sitzungen ist nicht erforderlich.
\end{enumerate}

\section{Fachschaftsfinanzen}
\label{sec:finanzen}
\begin{enumerate}
	\item \label{item:finanzen_beschluss}
Die Finanzräte dürfen nur dann Finanzmittel beim AStA beantragen, wenn zuvor ein Gremium der Fachschaft Physik die Ausgaben bewilligt hat.
	Von dieser Regel ausgenommen sind Ausgaben für die Durchführung des Sommerfests, der Orientierungswoche und der Erstsemester-Fahrt.
	\item Bei Ausgaben von Beträgen unter 10\,€ kann abweichend von \ref{sec:finanzen} Abs.~\ref{item:finanzen_beschluss} auf einen Beschluss eines Gremiums der Fachschaft Physik verzichtet werden.
	\item Die Finanzräte erstatten dem Fachschaftsrat \replaced{und der Fachschaftsvertretung monatlich Bericht über die Fachschaftsfinazen. Der Finanzbericht ist öffentlich.}{monatlich über die Ausgaben des vergangen Monats Bericht.}
\end{enumerate}

\section{Fachschaftsraum}
\begin{enumerate}
	\item Der FSR ist für die Ordnung und Sauberkeit des Fachschaftsraums zuständig.
	\item Zu Beginn eines Semesters wird eine Reinigungsliste angelegt, auf der festgehalten wird, wer in welcher Woche den Fachschaftsraum zu säubern hat.
	Jedes FSR-Mitglied ist dazu angehalten, mindestens einmal im Semester diesen Dienst zu übernehmen.
	\item Der Fachschaftsraum muss immer verschlossen sein, wenn niemand im Raum ist.
	\item Ein Schlüssel wird nur an diejenigen Personen ausgegeben, die diesen nachvollziehbar benötigen.
	Gleichzeitig wird erwartet, dass alle Mitglieder einen Präsenzdienst übernehmen, wenn ein Schlüssel ausgegeben wurde.
	\item Der Schlüssel muss spätestens mit Austritt aus dem FSR zurückgegeben werden.
	\item Auf Antrag kann der Schlüssel in Ausnahmefällen auch an ehemalige Mitglieder des FSR ausgegeben werden.
	Es ist zu begründen, wofür \added{und wie lange} dieser benötigt wird.
	\replaced{Der FSR entscheidet mit einfacher Mehrheit über den Antrag}{Eine Abstimmung auf einer FSR-Sitzung ist dazu erforderlich.}
	\item \added{er Fachschaftsraum und der Fachschafts-PC dient der Arbeit des FSR. Die Arbeit des FSR darf nicht durch nicht-fachschaftsbezogene Tätigkeiten beeinträchtigt werden.}
	\item \added{Übernachtungen im Fachschaftsraum sind verboten.}
\end{enumerate}

\section{\deleted{E-Mail-Verteiler und }Nutzergruppe}
\begin{enumerate}
	\item \deleted{Der FSR unterhält einen E-Mail-Verteiler: \email{fsphys-l@listserv.uni-muenster.de}.
	Die interne Kommunikation des FSR erfolgt über den Verteiler.}
	\item \deleted{Mitglied in dem E-Mail-Verteiler wird jedes Mitglied des FSR.
	Die Aufnahme weiterer Personen ist möglich.}
	\item \deleted{Mit Austritt aus dem FSR endet die Mitgliedschaft im E-Mail-Verteiler.
	Ehemalige Mitglieder des E-Mail-Verteilers können (auch mündlich) beantragen, für ein weiteres Jahr Mitglied im E-Mail-Verteiler zu sein.
	Mehrfachantrag ist möglich.
	Der FSR entscheidet über den Antrag.}
	\item Der FSR unterhält eine Nutzergruppe: \texttt{p0fsphys}.
	\deleted{Diese wird von Herrn Dr.~Berkemeier (\email{j.berkemeier@uni-muenster.de}) verwaltet.}
	Die Mitgliedschaft ermöglicht u.\,a.\ den Zugang zum Gruppenlaufwerk und das Drucken auf dem Drucker im Fachschaftsraum.
	\item Mitglied in der Nutzergruppe wird jedes Mitglied des FSR.
	Anträge auf Aufnahme und Verlängerung der Mitgliedschaft in der Nutzergruppe erfolgen über das Zentrum für Informationsverarbeitung (ZIV) und werden an den Verwalter der Nutzergruppe weitergeleitet.
	\item Mit Austritt aus dem FSR erlischt die Mitgliedschaft in der Nutzergruppe.
	Nur aus triftigem Grund ist eine weitere Mitgliedschaft möglich.
	Dies muss in jedem Einzelfall vom FSR beschlossen werden.
	\item Der FSR beauftragt eine oder mehrere Personen mit der Betreuung und Dokumentation \deleted{des E-Mail-Verteilers und }der Nutzergruppe.
	Diese Personen sollen in regelmäßigen Abständen, z.\,B.\ einmal im Jahr, die aktuellen Mitglieder ermitteln.
	Die Beauftragten für die Nutzergruppe sind Ansprechpartner für deren Verwalter.
\end{enumerate}

\section{Fachschaftsvollversammlung}
Für die Fachschaftsvollversammlung~(FVV) ist die Geschäftsordnung des Fachschaftsrats Physik anzuwenden, insbesondere im Hinblick auf Redeleitung und Protokollführung.

\section{Inkrafttreten}
Die Fachschaftsordnung der Fachschaft Physik der Westfälischen Wilhelms-Universität Münster wird durch Votum mit Zweidrittelmehrheit der Mitglieder der Fachschaftsvertretung beschlossen und tritt durch öffentlichen Aushang am \DTMdate{2018-01-10} am Folgetag in Kraft.

\end{document}
