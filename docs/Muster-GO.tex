%%% Pakete & Klassen
% Verwendung von KOMA-Script
\documentclass[
	% Papierformat
	a4paper,
	% Schriftgröße
	12pt,
	% einseitiges Layout
	oneside,
	% Abstand zwischen Absätzen statt Einrücken
	parskip=half-,
	% Linie unter der Kopfzeile
	headsepline,
	% deutsches Dokument (neue deutsche Rechtschreibung)
	ngerman,
	% Kleinere Seitenränder
	DIV=15
]{scrartcl}
\usepackage{iftex}
\ifLuaTeX
	% Einstellungen für Schriftart
	\usepackage{fontspec}
	% Silbentrennung, sprachspezifische Einstellungen
	\usepackage{polyglossia}
	\setmainlanguage{german}
	\usepackage{selnolig}
\else
	% Silbentrennung, sprachspezifische Einstellungen
	\usepackage{babel}
	% Mögliche darstellbare Zeichen (Umlaute, Sonderzeichen…)
	\usepackage[T1]{fontenc}
	% Zeichenkodierung der TeX-Datei
	\usepackage[utf8]{inputenc}
	% Schriftart
	\usepackage{lmodern}
	% führt Befehle für Sonderzeichen ein
	\usepackage{textcomp}
\fi
% Besseres Schriftbild (Mikrotypographie)
\usepackage{microtype}
% Für Kopf-/Fußzeile etc.
\usepackage{scrlayer-scrpage}
% Farben ermöglichen
\usepackage{xcolor}
% Paket für Bilder-Einbindung (EPS, PNG, JPG, PDF)
\usepackage{graphicx}
% Größere Freiheiten bei Dateinamen mit \includegraphics
\usepackage{grffile}
% Formatierung von Daten
\usepackage[useregional]{datetime2}
% Einstellungen für Aufzählungen
\usepackage{enumitem}
% „Schlaue“ Anführungszeichen
\usepackage{csquotes}

% Verlinkung, Querverweise können im PDF angeklickt werden
\usepackage[unicode]{hyperref}

%%% Einstellungen
% Anführungszeichen automatisch umwandeln
\MakeOuterQuote{"}

\hypersetup{
	% Links/Verweise in PDF mit Kasten der Dicke 0.5pt versehen
	pdfborder={0 0 0.5},
}

% Darstellung von sections anpassen
\renewcommand{\thesection}{§~\arabic{section}}

\titlehead{\vspace*{-2cm}%
	\centering
	\includegraphics[width=0.8\textwidth]{fs-physik-logo_v2024}%
}

%%% Neue Befehle
% Befehl zur Darstellung von E-Mail-Adressen
\newcommand{\email}[1]{\href{mailto:#1}{\texttt{#1}}}
% Semantischer Befehl für starke Betonung


% PDF-Metadaten
\hypersetup{
	pdfauthor={Fachschaftsrat Physik der Universität Münster},
	pdftitle={Muster-Geschäftsordnung der Fachschaft Physik der Universität Münster},
	pdfkeywords={Fachschaft, FSR, FSV, WWU, Münster, Geschäftsordnung}
}

% Kopfzeile
\ihead{Muster-Geschäftsordnung der FS Physik der Universität Münster}
\pagestyle{scrheadings}

%%% Bestandteile des Titels
\title{Muster-Geschäftsordnung}
\subtitle{der Fachschaft Physik\\
der Universität Münster}
\date{in der Fassung vom \DTMdate{2024-11-27}}
\author{}

\begin{document}

% Titel mit Logo
\maketitle

\section{Allgemeines}
Das Logo der Fachschaft Physik ist auf allen offiziellen Schriftstücken (Briefe, Poster, Publikationen und Ähnlichem) zu verwenden.
Beispielhaft ist dieses auf der ersten Seite der Geschäftsordnung aufgeführt.

\section{Einberufung und Vorbereitung der Sitzungen}
\begin{enumerate}
	\item Der Termin für die Sitzungen wird zu Beginn des Semesters von den Gremienmitgliedern festgelegt.
	\item In der vorlesungsfreien Zeit finden Sitzungen nur bei konkretem Bedarf statt.
	Sie werden mindestens eine Woche vorher durch den E-Mail-Verteiler oder Mattermost angekündigt.
	\item Außerordentliche Sitzungen müssen mindestens zwei Tage vorher aus triftigem Grund einberufen werden.
	\item Die Sitzungen sind öffentlich.
	Auf Antrag eines Mitglieds und mit einfacher Mehrheit aller anwesenden stimmberechtigten Mitglieder ist die Sitzung oder sind einzelne Tagesordnungspunkte nicht öffentlich.
	\item Alle Mitglieder sollten möglichst zu jedem Termin erscheinen.
	Abwesenheit sollte zuvor angemeldet werden.
	\item Abstimmungen sind nur unter einem eigenen Tagesordnungspunkt möglich.
\end{enumerate}

\section{Durchführung der Sitzungen}
\label{sec:Sitzungen}
\begin{enumerate}
	\item Die Eröffnung der Sitzung obliegt dem Vorsitz oder dessen Vertretung.
	Sollten Vorsitz und Vertretung nicht anwesend sein, eröffnet das dienstälteste Mitglied die Sitzung.
	\item Zu Beginn der Sitzung werden folgende Dinge festgestellt bzw.\ zugewiesen: Beschlussfähigkeit, Redeleitung, Protokollführung, Richtigkeit alter Protokolle, so sie noch nicht genehmigt sind, und die Tagesordnung.
	\item Die Redeleitung obliegt dem Vorsitz oder der Vertretung.
	Sollten Vorsitz und Vertretung nicht anwesend sein, wird zu Beginn der Sitzung nach den unter \ref{sec:abstimmungen} definierten Regeln die Redeleitung gewählt.
	\item Die Protokollführung kann in einem rotierenden System festgelegt werden, wenn sich kein Mitglied bereit erklärt, dieses anzufertigen.
	Regelungen zur Protokollführung sind in \ref{sec:protokolle} zu finden.
	\item Auf Antrag kann die Redezeit auf drei Minuten begrenzt werden.
	\item Bei diskriminierenden Aussprüchen oder Redeinhalten sowie persönlichen Beleidigungen behält sich das Gremium nach Mehrheitsbeschluss Sanktionen vor.
	Diese können von einer Verwarnung über ein Redeverbot bis zum Verweis aus dem Sitzungsraum gehen.
	\item Wortmeldungen zur Geschäftsordnung (GO) gehen allen anderen Wortmeldungen vor.
	Auch auf einen solchen Antrag darf das Wort jedoch nicht erteilt werden, solange eine Person redet, der die Redeleitung zur Zeit der Antragsstellung das Wort bereits erteilt hatte, oder solange eine Wahl oder Abstimmung läuft, deren Beginn die Redeleitung vor der Wortmeldung festgestellt hatte.
	\item Als Anträge zur GO sind insbesondere anzusehen Anträge auf:
	\begin{enumerate}
		\item Schluss der Redeliste.
		Jedoch nur von Personen, die selbst nicht zur Sache gesprochen haben.
		\item Schluss der Aussprache, ggf.\ sofortige Abstimmung.
		Jedoch nur von Personen, die selbst nicht zur Sache gesprochen haben.
		\item Vertagung der Beschlussfassung über einen Antrag.
		\item Vertagung eines Punktes der Tagesordnung.
		\item Nichtbefassung mit einem Tagesordnungspunkt (TOP) oder Antrag.
		\item Unterbrechung der Sitzung.
		\item Feststellung der Beschlussfähigkeit.
		\item Sofortige Wiederholung einer Abstimmung oder eines Wahlganges wegen offensichtlicher Formfehler oder wegen objektiver Unklarheit über den Inhalt oder die Abstimmung.
		\item Schluss der Sitzung (Zweidrittelmehrheit notwendig).
		\item Zurückkommen auf einen bereits abgeschlossenen TOP (Zweidrittelmehrheit notwendig).
		\item Änderung der Tagesordnung.
	\end{enumerate}
	\item Ein Antrag zur Geschäftsordnung gilt als angenommen, wenn ihm nicht widersprochen wird.
	Bei Widerspruch ist nach der Anhörung von höchstens je einem Redebeitrag für und gegen den Antrag abzustimmen.
	Begründungspflicht besteht bei Widerspruch nicht (formale Ablehnung).
	Gegen einen Antrag auf Feststellung der Beschlussfähigkeit kann kein Widerspruch eingelegt werden.
\end{enumerate}

\section{Abstimmungen}
\label{sec:abstimmungen}
\begin{enumerate}
	\item Jedes Mitglied besitzt eine Stimme.
	Stimmberechtigt sind alle gewählten Mitglieder.
	\item \label{item:Sitzungsbeginn} Kann auf einer Sitzung über Anträge, die aufgrund ihrer Dringlichkeit nicht auf die nächste Sitzung verschoben werden können, wegen Beschlussunfähigkeit nicht abgestimmt werden, so sind diese Anträge schriftlich abzustimmen.
	\item Ist ein TOP zur Entscheidung reif, so eröffnet die Redeleitung nach Abfragen der Anträge die Abstimmung.
	Anträge zum Abstimmungsgegenstand sind von diesem Zeitpunkt an nicht mehr zulässig.
	Das Recht auf anschließende Anträge zur Geschäftsordnung bleibt unberührt.
	\item Über weitergehende Anträge wird zuerst abgestimmt.
	Falls es inhaltlich nicht zu klären ist, entscheidet die Reihenfolge der Antragsstellung.
	\item Änderungsvorschläge zu einem Sachantrag sind vor dem Hauptantrag zur Abstimmung zu stellen.
	Soweit das Gremium den Änderungsanträgen zustimmt oder sie von der Antragstellerin oder dem Antragsteller übernommen werden, wird der Hauptantrag in der geänderten Fassung zur Abstimmung gestellt.
	\item Im Normalfall wird in der Sitzung durch Handzeichen abgestimmt.
	Auf Antrag einer stimmberechtigten Person muss eine geheime Abstimmung durchgeführt werden.
	\item Ein Antrag, über den mittels Mehrheitsabstimmung zu entscheiden ist, wird (falls nicht anders durch die GO geregelt) bei einer einfachen Mehrheit von Ja-Stimmen angenommen. Bei gleich vielen Ja- und Nein-Stimmen gilt ein Antrag als abgelehnt. Liegen mehr Enthaltungen als Ja-Stimmen vor, muss die Abstimmung einmal wiederholt werden. Bei einer wiederholten Abstimmung ist ein Antrag unabhängig von der Zahl der Enthaltungen angenommen, wenn mehr Ja- als Nein-Stimmen vorliegen. Enthält ein Hauptantrag einen Teilantrag, über den mittels Alternativabstimmung zu entscheiden ist, gilt die Option als angenommen, welche eine absolute Mehrheit an Stimmen auf sich vereint. Erhält keine der Optionen eine absolute Mehrheit, erfolgt eine Stichwahl zwischen den beiden Optionen mit den meisten Stimmen (beziehungsweise allen weiteren Optionen, die gleich viele Stimmen erhalten haben). Die Regelung bezüglich Enthaltungen gilt analog zum Abstimmungsverfahren bei Mehrheitsabstimmungen.
	\label{item:abstimmungsmodus}
\end{enumerate}

\section{Schriftliche Abstimmungen}
\label{sec:schriftlich}
\begin{enumerate}
	\item Wird eine geheime Abstimmung gewünscht, so kann die Abstimmung nicht schriftlich stattfinden.
	In dringenden Fällen ist eine Sondersitzung einzuberufen.
	\item Der Vorsitz hat bei jeder schriftlichen Abstimmung dafür Sorge zu tragen, dass die zur Abstimmung stehenden Anträge sowie die Art der Stimmabgabe klar erkennbar sind.
	Die Willensäußerungen der Mitglieder des Gremiums müssen ihre Haltung zum verlangten Beschluss eindeutig erkennen lassen.
	\item Die Willensäußerungen müssen mindestens bis zur nächsten regulären Sitzung archiviert, auf Verlangen eines Mitglieds vorgelegt und dem Protokoll der nächsten regulären Sitzung beigefügt werden.
	\item Das Verfahren
	der schriftlichen Abstimmung gilt als abgeschlossen, wenn die für die Annahme oder
	Ablehnung des zur Abstimmung gestellten Antrags erforderliche Mehrheit der Stimmen
	aller Mitglieder erreicht ist oder die von der antragstellenden Person angesetzte Frist, die
	mindestens drei Tage zu betragen hat, abgelaufen ist.
	\item Bei einer schriftlichen Abstimmung ist ein Antrag unabhängig von der Zahl der Enthaltungen angenommen, wenn mehr Ja- als Nein-Stimmen vorliegen.
	\item Schriftliche Abstimmungen finden per E-Mail bzw. Mattermost statt.
\end{enumerate}

\section{Protokolle}
\label{sec:protokolle}
\begin{enumerate}
	\item Von jeder Sitzung wird ein Protokoll erstellt.
	\item Auf Beschluss kann das Protokoll einen nicht-öffentlichen Teil enthalten.
	Dies ist insbesondere dann zu wählen, wenn mit der Nennung Persönlichkeitsrechte verletzt werden können.
	\item Das Protokoll sollte in elektronischer Form (\LaTeX) angefertigt werden, um eine gut lesbare, dauerhafte und durchsuchbare Dokumentation zu erhalten.
	\item Die öffentliche und nicht-öffentliche Version des Protokolls wird auf dem Netzlaufwerk der Fachschaft archiviert.
	Eine Veröffentlichung auf den Internetseiten der Fachschaft ist möglich.
	\item Auf Antrag kann eine Kopie des öffentlichen Protokolls erstellt und ausgehändigt werden.
	\item Das Protokoll gilt als genehmigt, wenn auf der folgenden Sitzung gemäß \ref{sec:Sitzungen} Abs.~\ref{item:Sitzungsbeginn} keine Gegenrede erhoben wird.
	\item Protokolle enthalten mindestens
	\begin{itemize}
		\item Beginn und Ende der Sitzung,
		\item Anwesende auf der Sitzung und gegebenenfalls deren verspätetes Eintreffen oder vorzeitiges Verlassen der Sitzung,
		\item Antragstexte oder eindeutige Verweise auf die Anträge,
		\item Abstimmungsergebnisse,
		\item Anträge zur Geschäftsordnung und deren Behandlung,
		\item Sondervoten.
	\end{itemize}
\end{enumerate}

\end{document}
