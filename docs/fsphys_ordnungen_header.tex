%%% Pakete & Klassen
% Verwendung von KOMA-Script
\documentclass[
	% Papierformat
	a4paper,
	% Schriftgröße
	12pt,
	% einseitiges Layout
	oneside,
	% Abstand zwischen Absätzen statt Einrücken
	parskip=half-,
	% Linie unter der Kopfzeile
	headsepline,
	% deutsches Dokument (neue deutsche Rechtschreibung)
	ngerman,
	% Kleinere Seitenränder
	DIV=15
]{scrartcl}
\usepackage{iftex}
\ifLuaTeX
	% Einstellungen für Schriftart
	\usepackage{fontspec}
	% Silbentrennung, sprachspezifische Einstellungen
	\usepackage{polyglossia}
	\setmainlanguage{german}
	\usepackage{selnolig}
\else
	% Silbentrennung, sprachspezifische Einstellungen
	\usepackage{babel}
	% Mögliche darstellbare Zeichen (Umlaute, Sonderzeichen…)
	\usepackage[T1]{fontenc}
	% Zeichenkodierung der TeX-Datei
	\usepackage[utf8]{inputenc}
	% Schriftart
	\usepackage{lmodern}
	% führt Befehle für Sonderzeichen ein
	\usepackage{textcomp}
\fi
% Besseres Schriftbild (Mikrotypographie)
\usepackage{microtype}
% Für Kopf-/Fußzeile etc.
\usepackage{scrlayer-scrpage}
% Farben ermöglichen
\usepackage{xcolor}
% Paket für Bilder-Einbindung (EPS, PNG, JPG, PDF)
\usepackage{graphicx}
% Größere Freiheiten bei Dateinamen mit \includegraphics
\usepackage{grffile}
% Formatierung von Daten
\usepackage[useregional]{datetime2}
% Einstellungen für Aufzählungen
\usepackage{enumitem}
% „Schlaue“ Anführungszeichen
\usepackage{csquotes}

% Verlinkung, Querverweise können im PDF angeklickt werden
\usepackage[unicode]{hyperref}

%%% Einstellungen
% Anführungszeichen automatisch umwandeln
\MakeOuterQuote{"}

\hypersetup{
	% Links/Verweise in PDF mit Kasten der Dicke 0.5pt versehen
	pdfborder={0 0 0.5},
}

% Darstellung von sections anpassen
\renewcommand{\thesection}{§~\arabic{section}}

\titlehead{\vspace*{-2cm}%
	\centering
	\includegraphics[width=0.8\textwidth]{fs-physik-logo_v2024}%
}

%%% Neue Befehle
% Befehl zur Darstellung von E-Mail-Adressen
\newcommand{\email}[1]{\href{mailto:#1}{\texttt{#1}}}
% Semantischer Befehl für starke Betonung
