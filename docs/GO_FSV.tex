%%% Pakete & Klassen
% Verwendung von KOMA-Script
\documentclass[
	% Papierformat
	a4paper,
	% Schriftgröße
	12pt,
	% einseitiges Layout
	oneside,
	% Abstand zwischen Absätzen statt Einrücken
	parskip=half-,
	% Linie unter der Kopfzeile
	headsepline,
	% deutsches Dokument (neue deutsche Rechtschreibung)
	ngerman,
	% Kleinere Seitenränder
	DIV=15
]{scrartcl}
\usepackage{iftex}
\ifLuaTeX
	% Einstellungen für Schriftart
	\usepackage{fontspec}
	% Silbentrennung, sprachspezifische Einstellungen
	\usepackage{polyglossia}
	\setmainlanguage{german}
	\usepackage{selnolig}
\else
	% Silbentrennung, sprachspezifische Einstellungen
	\usepackage{babel}
	% Mögliche darstellbare Zeichen (Umlaute, Sonderzeichen…)
	\usepackage[T1]{fontenc}
	% Zeichenkodierung der TeX-Datei
	\usepackage[utf8]{inputenc}
	% Schriftart
	\usepackage{lmodern}
	% führt Befehle für Sonderzeichen ein
	\usepackage{textcomp}
\fi
% Besseres Schriftbild (Mikrotypographie)
\usepackage{microtype}
% Für Kopf-/Fußzeile etc.
\usepackage{scrlayer-scrpage}
% Farben ermöglichen
\usepackage{xcolor}
% Paket für Bilder-Einbindung (EPS, PNG, JPG, PDF)
\usepackage{graphicx}
% Größere Freiheiten bei Dateinamen mit \includegraphics
\usepackage{grffile}
% Formatierung von Daten
\usepackage[useregional]{datetime2}
% Einstellungen für Aufzählungen
\usepackage{enumitem}
% „Schlaue“ Anführungszeichen
\usepackage{csquotes}

% Verlinkung, Querverweise können im PDF angeklickt werden
\usepackage[unicode]{hyperref}

%%% Einstellungen
% Anführungszeichen automatisch umwandeln
\MakeOuterQuote{"}

\hypersetup{
	% Links/Verweise in PDF mit Kasten der Dicke 0.5pt versehen
	pdfborder={0 0 0.5},
}

% Darstellung von sections anpassen
\renewcommand{\thesection}{§~\arabic{section}}

\titlehead{\vspace*{-2cm}%
	\centering
	\includegraphics[width=0.8\textwidth]{fs-physik-logo_v2024}%
}

%%% Neue Befehle
% Befehl zur Darstellung von E-Mail-Adressen
\newcommand{\email}[1]{\href{mailto:#1}{\texttt{#1}}}
% Semantischer Befehl für starke Betonung


% PDF-Metadaten
\hypersetup{
	pdfauthor={Fachschaftsvertretung Physik der WWU Münster},
	pdftitle={Geschäftsordnung der Fachschaftsvertretung Physik der Westfälischen Wilhelms-Universität Münster},
	pdfkeywords={Fachschaftsvertretung, FSV, WWU, Münster, Geschäftsordnung}
}

% Kopfzeile
\ihead{Geschäftsordnung der FSV Physik der Westfälischen Wilhelms-Universität Münster}
\pagestyle{scrheadings}

%%% Bestandteile des Titels
\title{Geschäftsordnung}
\subtitle{der Fachschaftsvertretung Physik\\
der Westfälischen Wilhelms-Universität Münster}
\date{in der Fassung vom \DTMdate{2017-07-26}}
\author{}

\begin{document}

% Titel mit Logo
\maketitle

\section{Einberufung der Sitzungen der Fachschaftsvertretung}
\begin{enumerate}
	\item Jedes Mitglied der Fachschaftsvetrertung (FSV) kann eine FSV-Sitzung einberufen.
	\item Eine Sitzung der FSV muss mindestens eine Woche zuvor durch Benachrichtigung aller Mitglieder der FSV einberufen werden.
\end{enumerate}

\section{Durchführung der Sitzungen der Fachschaftsvertretung}
\label{sec:sitzung_durchfuehrung}
\begin{enumerate}
	\item Die Eröffnung der Sitzung obliegt dem Mitglied der FSV, das die Sitzung einberufen hat.
	\item Zu Beginn der Sitzung werden folgende Dinge in folgender Reihenfolge festgestellt bzw.\ (durch Wahl) zugewiesen:
	\begin{enumerate}
		\item Beschlussfähigkeit
		\item Redeleitung
		\item Protokollant
		\item Tagesordnung
	\end{enumerate}
	\item Bei diskriminierenden Aussprüchen oder Redeinhalten sowie persönlichen Beleidigungen behält sich die FSV nach Mehrheitsbeschluss Sanktionen vor.
	Diese können von einer Verwarnung über ein Redeverbot bis zum Verweis aus dem Sitzungsraum gehen.
	\item Wortmeldungen zur Geschäftsordnung (GO) gehen allen anderen Wortmeldungen vor.
	Auch auf einen solchen Antrag darf das Wort jedoch nicht erteilt werden, solange eine Person redet, der die Redeleitung zur Zeit der Antragsstellung das Wort bereits erteilt hatte, oder solange eine Wahl oder Abstimmung läuft, deren Beginn die Redeleitung vor der Wortmeldung festgestellt hatte.
	\item \label{item:go_antraege}
	Anträge zur GO sind Anträge auf:
	\begin{enumerate}
		\item Schluss der Redeliste.
		Jedoch nur von Personen, die selbst nicht zur Sache gesprochen haben.
		\item Schluss der Aussprache, ggf.\ sofortige Abstimmung.
		Jedoch nur von Personen, die selbst nicht zur Sache gesprochen haben.
		\item Vertagung der Beschlussfassung über einen Antrag.
		\item Vertagung eines Punkts der Tagesordnung.
		\item Nichtbefassung mit einem Tagesordnungspunkt (TOP) oder Antrag.
		\item Unterbrechung der Sitzung.
		\item Sofortige Wiederholung einer Abstimmung oder eines Wahlgangs wegen offensichtlicher Formfehler oder wegen objektiver Unklarheit über den Inhalt oder die Abstimmung.
		\item Änderung der Tagesordnung.
		\item Feststellung der Beschlussfähigkeit (Widerspruch nicht möglich).
		\item Geheime Wahl oder Abstimmung (Widerspruch nicht möglich).
		\item Schluss der Sitzung (Zweidrittelmehrheit notwendig).
		\item Zurückkommen auf einen bereits abgeschlossenen TOP (Zweidrittelmehrheit notwendig).
		\item Abweichung von den Bestimmungen dieser Geschäftsordnung (Zweidrittelmehrheit notwendig).
	\end{enumerate}
	\item Ein Antrag zur GO gilt als angenommen, wenn ihm nicht widersprochen wird.
	Bei Widerspruch ist nach der Anhörung von höchstens je einer Rednerin/einem Redner für und gegen den Antrag abzustimmen.
	Begründungspflicht besteht bei Widerspruch nicht (formale Ablehnung).
\end{enumerate}

\section{Abstimmungen}
\begin{enumerate}
	\item Jedes FSV-Mitglied besitzt eine Stimme.
	Stimmberechtigt sind alle gewählten Mitglieder der FSV.
	% Grund für die Streichung: Kopiert aus der Satzung der Studierendenschaft
	\item Ist ein TOP zur Entscheidung reif, so eröffnet die Redeleitung nach Abfragen der Anträge die Abstimmung.
	Anträge zum Abstimmungsgegenstand sind von diesem Zeitpunkt an nicht mehr zulässig.
	Das Recht auf anschließende Anträge zur Geschäftsordnung bleibt unberührt.
	\item Im Normalfall wird in der FSV-Sitzung durch Handzeichen abgestimmt.
	Auf Antrag einer stimmberechtigten Person muss gemäß \ref{sec:sitzung_durchfuehrung} Abs.~\ref{item:go_antraege} eine geheime Abstimmung durchgeführt werden.
	\item Ein Antrag wird (falls nicht anders durch die GO geregelt) bei einer einfachen Mehrheit von Ja-Stimmen angenommen.
	Bei einer Mehrheit von Nein-Stimmen oder gleich vielen Ja- und Nein-Stimmen gilt ein Antrag als abgelehnt.
	
	Liegen mehr Ja- als Nein-Stimmen, jedoch auch mehr Enthaltungen als Ja-Stimmen vor, muss die Abstimmung einmal wiederholt werden.
	Bei einer wiederholten Abstimmung ist ein Antrag unabhängig von der Zahl der Enthaltungen angenommen, wenn mehr Ja- als Nein-Stimmen vorliegen.
\end{enumerate}

\section{Wahlen}
\begin{enumerate}
	\item Jedes Mitglied der FSV kann sich und andere Mitglieder der Fachschaft Physik zur Wahl vorschlagen.
	\item Wahlen werden von der Redeleitung durchgeführt.
	Sollte die Redeleitung selbst zur Wahl stehen, so wird ein Wahlleiter für diese Wahl gewählt.
	% Grund für die Streichung: Kopiert aus der Fachschaftsordnung; außerdem nicht ganz korrekt
	% Grund für die Streichung:
	% 	- Erster Satz: Kopiert aus der Satzung der Studierendenschaft, s. § 7
	% 	- Rest: Obsolet durch neue Bestimmungen in der FO
	% Grund für die Streichung: Kopiert aus der Satzung der Studierendenschaft, s. § 7
\end{enumerate}

\section{Protokolle}
% Grund für die Streichung: Kopiert aus der Satzung der Studierendenschaft, s. § 10 Abs. 6
Die Protokolle der FSV-Sitzungen werden jedem Mitglied der FSV und dem FSR zur Verfügung gestellt.

% Grund für die Streichung: Die GO werden laut Satzung der Studierendenschaft mit absoluter Mehrheit beschlossen, s. § 8 Abs. 1

\section{Inkrafttreten}
Die Geschäftsordnung der Fachschaftsvertretung Physik der Westfälischen Wilhelms-Universität Münster tritt durch Beschluss der Fachschaftsvertretung und durch öffentlichen Aushang am \DTMdate{2017-07-26} in Kraft.

\end{document}
