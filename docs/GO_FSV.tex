%%% Pakete & Klassen
% Verwendung von KOMA-Script
\documentclass[
	% Papierformat
	a4paper,
	% Schriftgröße
	12pt,
	% einseitiges Layout
	oneside,
	% Abstand zwischen Absätzen statt Einrücken
	parskip=half-,
	% Linie unter der Kopfzeile
	headsepline,
	% deutsches Dokument (neue deutsche Rechtschreibung)
	ngerman,
	% Kleinere Seitenränder
	DIV=15
]{scrartcl}
\usepackage{iftex}
\ifLuaTeX
	% Einstellungen für Schriftart
	\usepackage{fontspec}
	% Silbentrennung, sprachspezifische Einstellungen
	\usepackage{polyglossia}
	\setmainlanguage{german}
	\usepackage{selnolig}
\else
	% Silbentrennung, sprachspezifische Einstellungen
	\usepackage{babel}
	% Mögliche darstellbare Zeichen (Umlaute, Sonderzeichen…)
	\usepackage[T1]{fontenc}
	% Zeichenkodierung der TeX-Datei
	\usepackage[utf8]{inputenc}
	% Schriftart
	\usepackage{lmodern}
	% führt Befehle für Sonderzeichen ein
	\usepackage{textcomp}
\fi
% Besseres Schriftbild (Mikrotypographie)
\usepackage{microtype}
% Für Kopf-/Fußzeile etc.
\usepackage{scrlayer-scrpage}
% Farben ermöglichen
\usepackage{xcolor}
% Paket für Bilder-Einbindung (EPS, PNG, JPG, PDF)
\usepackage{graphicx}
% Größere Freiheiten bei Dateinamen mit \includegraphics
\usepackage{grffile}
% Formatierung von Daten
\usepackage[useregional]{datetime2}
% Einstellungen für Aufzählungen
\usepackage{enumitem}
% „Schlaue“ Anführungszeichen
\usepackage{csquotes}

% Verlinkung, Querverweise können im PDF angeklickt werden
\usepackage[unicode]{hyperref}

%%% Einstellungen
% Anführungszeichen automatisch umwandeln
\MakeOuterQuote{"}

\hypersetup{
	% Links/Verweise in PDF mit Kasten der Dicke 0.5pt versehen
	pdfborder={0 0 0.5},
}

% Darstellung von sections anpassen
\renewcommand{\thesection}{§~\arabic{section}}

\titlehead{\vspace*{-2cm}%
	\centering
	\includegraphics[width=0.8\textwidth]{fs-physik-logo_v2024}%
}

%%% Neue Befehle
% Befehl zur Darstellung von E-Mail-Adressen
\newcommand{\email}[1]{\href{mailto:#1}{\texttt{#1}}}
% Semantischer Befehl für starke Betonung


% PDF-Metadaten
\hypersetup{
	pdfauthor={Fachschaftsvertretung Physik der WWU Münster},
	pdftitle={Geschäftsordnung der Fachschaftsvertretung Physik der Westfälischen Wilhelms-Universität Münster},
	pdfkeywords={Fachschaftsvertretung, FSV, WWU, Münster, Geschäftsordnung}
}

% Kopfzeile
\ihead{Geschäftsordnung der FSV Physik der Westfälischen Wilhelms-Universität Münster}
\pagestyle{scrheadings}

%%% Bestandteile des Titels
\title{Geschäftsordnung}
\subtitle{der Fachschaftsvertretung Physik\\
der Westfälischen Wilhelms-Universität Münster}
\date{in der Fassung vom \DTMdate{2016-01-06}}
\author{}

\begin{document}

% Titel mit Logo
\maketitle

\section{Einberufung der FSV-Sitzungen}
\begin{enumerate}
	\item Jedes Mitglied der Fachschaftsvetrertung (FSV) kann eine FSV-Sitzung einberufen.
	\item Eine Sitzung der FSV muss mindestens eine Woche zuvor durch Benachrichtigung aller Mitglieder der FSV einberufen werden.
\end{enumerate}

\section{Durchführung der FSV-Sitzungen}
\begin{enumerate}
	\item Die Eröffnung der Sitzung obliegt dem Mitglied der FSV, das die Sitzung einberufen hat.
	\item Zu Beginn der Sitzung werden folgende Dinge in folgender Reihenfolge festgestellt bzw.\ (durch Wahl) zugewiesen:
	\begin{enumerate}
		\item Beschlussfähigkeit
		\item Redeleitung
		\item Protokollant
		\item Tagesordnung
	\end{enumerate}
	\item Bei diskriminierenden Aussprüchen oder Redeinhalten sowie persönlichen Beleidigungen behält sich die FSV nach Mehrheitsbeschluss Sanktionen vor. Diese können von einer Verwarnung über ein Redeverbot bis zum Verweis aus dem Sitzungsraum gehen.
	\item Wortmeldungen zur Geschäftsordnung (GO) gehen allen anderen Wortmeldungen vor. Auch auf einen solchen Antrag darf das Wort jedoch nicht erteilt werden, solange eine Person redet, der die Redeleitung zur Zeit der Antragsstellung das Wort bereits erteilt hatte, oder solange eine Wahl oder Abstimmung läuft, deren Beginn die Redeleitung vor der Wortmeldung festgestellt hatte.
	\item Als Anträge zur GO sind insbesondere anzusehen Anträge auf:
	\begin{enumerate}
		\item Schluss der Redeliste. Jedoch nur von Personen, die selbst nicht zur Sache gesprochen haben.
		\item Schluss der Aussprache, ggf.\ sofortige Abstimmung. Jedoch nur von Personen, die selbst nicht zur Sache gesprochen haben.
		\item Vertagung der Beschlussfassung über einen Antrag.
		\item Vertagung eines Punktes der Tagesordnung.
		\item Nichtbefassung mit einem Tagesordnungspunkt (TOP) oder Antrag.
		\item Unterbrechung der Sitzung.
		\item Feststellung der Beschlussfähigkeit.
		\item Sofortige Wiederholung einer Abstimmung oder eines Wahlganges wegen offensichtlicher Formfehler oder wegen objektiver Unklarheit über den Inhalt oder die Abstimmung.
		\item Schluss der Sitzung (Zweidrittelmehrheit notwendig).
		\item Zurückkommen auf einen bereits abgeschlossenen TOP (Zweidrittelmehrheit notwendig).
		\item Änderung der Tagesordnung.
	\end{enumerate}
	\item Ein Antrag zur Geschäftsordnung gilt als angenommen, wenn ihm nicht widersprochen wird. Bei Widerspruch ist nach der Anhörung von höchstens je einer Rednerin/einem Redner für und gegen den Antrag abzustimmen. Begründungspflicht besteht bei Widerspruch nicht (formale Ablehnung).
\end{enumerate}

\section{Abstimmungen}
\label{sec:abstimmungen}
\begin{enumerate}
	\item Jedes FSV-Mitglied besitzt eine Stimme. Stimmberechtigt sind alle gewählten Mitglieder der FSV.
	\item Gemäß §~10 Abs.~2 der Satzung der Studierendenschaft ist die FSV beschlussfähig, wenn mindestens die Hälfte aller FSV-Mitglieder anwesend ist. Außerdem gilt demnach die Beschlussfähigkeit so lange als gegeben, bis auf Antrag eines Mitglieds die Beschlussunfähigkeit formell festgestellt wird.
	\item Ist ein TOP zur Entscheidung reif, so eröffnet die Redeleitung nach Abfragen der Anträge die Abstimmung. Anträge zum Abstimmungsgegenstand sind von diesem Zeitpunkt an nicht mehr zulässig. Das Recht auf anschließende Anträge zur Geschäftsordnung bleibt unberührt.
	\item Im Normalfall wird in der FSV-Sitzung durch Handzeichen abgestimmt. Auf Antrag einer stimmberechtigten Person muss eine geheime Abstimmung durchgeführt werden.
	\item Ein Antrag wird (falls nicht anders durch die GO geregelt) bei einer einfachen Mehrheit von Ja-Stimmen angenommen. Bei einer Mehrheit von Nein-Stimmen oder gleich vielen Ja- und Nein-Stimmen gilt ein Antrag als abgelehnt.
	
	Liegen mehr Ja- als Nein-Stimmen, jedoch auch mehr Enthaltungen als Ja-Stimmen vor, muss die Abstimmung einmal wiederholt werden. Bei einer wiederholten Abstimmung ist ein Antrag unabhängig von der Zahl der Enthaltungen angenommen, wenn mehr Ja- als Nein-Stimmen vorliegen.
\end{enumerate}

\section{Wahlen}
\begin{enumerate}
	\item Jedes Mitglied der FSV kann sich und andere Mitglieder der Fachschaft Physik zur Wahl vorschlagen.
	\item Wahlen werden von der Redeleitung durchgeführt. Sollte die Redeleitung selbst zur Wahl stehen, so wird ein Wahlleiter für diese Wahl gewählt.
	\item Gemäß §~3 Abs.~2 der Fachschaftsordnung der Fachschaft Physik erfolgt die Wahl des Fachschaftsrats (FSR) mit Handzeichen unter Ausschluss der Öffentlichkeit. Auf Antrag einer anwesenden wahlberechtigten Person muss eine geheime Wahl stattfinden. Es kann auch über eine Liste als Ganzes abgestimmt werden. 
	\item Bei Personenwahlen können die Mitglieder der FSV für einen Kandidaten stimmen oder sich der Stimme enthalten. Bei Listenwahlen können die Mitglieder der FSV für eine Vorschlagsliste stimmen oder sich ihrer Stimme enthalten. Steht nur eine Liste zur Wahl, so können die Mitglieder der FSV zusätzlich gegen die Vorschlagsliste stimmen. In diesem Fall gelten die in §~3 Abs.~5 beschrieben Kriterien über die Wahl der Liste.
	\item Durch Personenwahl ist gewählt, wer eine absolute Mehrheit erreicht. Ergibt sich im ersten Wahlgang keine absolute Mehrheit, so folgt ein zweiter Wahlgang. Ergibt sich auch im zweiten Wahlgang keine absolute Mehrheit, findet ein dritter Wahlgang statt. Im dritten Wahlgang ist gewählt, wer eine relative Mehrheit erhält. Vereinigen mehrere Kandidaten im dritten Wahlgang gleich viele und jeweils die meisten Stimmen auf sich, findet eine Stichwahl zwischen diesen statt. Vereinigen auch in der Stichwahl mehrere Kandidaten gleich viele und jeweils die meisten Stimmen auf sich, entscheidet zwischen ihnen das Los. 
\end{enumerate}

\section{Protokolle}
\begin{enumerate}
	\item Von jeder Sitzung wird ein Protokoll durch den gewählten Protokollanten erstellt.
	\item Das Protokoll wird jedem Mitglied der FSV und dem FSR zur Verfügung gestellt. 
\end{enumerate}

\section{Beschluss und Änderung der Geschäftsordnung}
Die Geschäftsordnung und deren Änderungen können nur mit Zweidrittelmehrheit aller FSV-Mitglieder beschlossen werden.

\section{Inkrafttreten}
Die Geschäftsordnung der Fachschaftsvertretung Physik der Westfälischen Wilhelms-Universität Münster tritt durch Beschluss der Fachschaftsvertretung und durch öffentlichen Aushang am \DTMdate{2016-01-07} in Kraft.

\end{document}
